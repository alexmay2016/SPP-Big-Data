\documentclass[12pt]{article}
\usepackage[a4paper,hmargin=1in,vmargin=1.25in]{geometry}

%\usepackage{german}
\usepackage{pgfgantt} % Gantt-Chart
\usepackage{adjustbox} 	% to resize a object to textwidth
\usepackage{pdflscape}  % to use \begin{landscape} testText \end{landscape}
\usepackage{latexsym,wrapfig}
\usepackage{amsmath,amssymb}
\usepackage{epsfig,color,colordvi}
\usepackage{graphics}
\usepackage{theorem}
\usepackage{pstricks}
\usepackage{amsfonts}
\usepackage[T1]{fontenc}
\usepackage{eurosym}

%\textwidth 16.1cm      %changea4.tex
%\textheight 170mm      %
%\textheight 223mm      %
%\evensidemargin 0mm    %
%\oddsidemargin 0mm     %
%\topmargin -7mm

\parskip1ex

\pagestyle{plain}

\input epsf

\date{\today}
\newcommand{\mmod}{\hspace{1mm}{\rm mod}\hspace{1mm}}
\newcommand{\lf}{\left\lfloor}
\newcommand{\rf}{\right\rfloor}
\newcommand{\norm}{|\!|}
\newcommand{\Q}{\mathbb{Q}}
\newcommand{\N}{\mathbb{N}}
\newcommand{\C}{\mathbb{C}}
\newcommand{\Z}{\mathbb{Z}}
\newcommand{\R}{\mathbb{R}}
\newcommand{\F}{\mathbb{F}}
\newcommand{\mb}{\mathbf}
\newcommand{\bigO}{{\cal O}}
\newcommand{\res}{\textrm{res}}
\newcommand{\poly}{\textrm{poly}}
\DeclareMathOperator{\wt}{wt}

\newbox\BeweisSym
\setbox\BeweisSym=\hbox{\unitlength=0.18ex%
\begin{picture}(10,10)
\put(0,0){\framebox(9,9){}}
\put(0,3){\framebox(6,6){}}
\end{picture}}
%
\newenvironment{Proof}{\noindent{\bf Proof:}$\mbox{}\;$}%
{\hfill\copy\BeweisSym\linebreak\par\noindent}
\newtheorem{Claim}{Claim}
\renewcommand{\indexname}{Was Immer Da stehen soll}

\newcommand{\marc}[1]{\textcolor{red}{\texttt{Marc: }#1}}
\newcommand{\alex}[1]{\textcolor{red}{\texttt{Alex: }#1}}
\newcommand{\map}{\textsf{map}}
\newcommand{\reduce}{\textsf{reduce}}
\newcommand{\adv}{\ensuremath{\mathcal{A}}}
\newcommand{\reduc}{\ensuremath{\mathcal{R}}}
\newcommand{\memoc}{\text{MemoC}}

\newcommand{\url}[1]{\texttt{#1}}

\begin{document}

\noindent
{\large
Project Proposal\\
within SPP 1736 -- Algorithms for Big Data\\[2ex]
\textbf{Memory-Efficiency for Big Data Cryptography (MemoC)}\\
}

\noindent
Marc Fischlin, TU Darmstadt\\
Alexander May, Ruhr Universit\"at Bochum\\

\noindent
{\hrulefill}

\iffalse
\subsection{Applicants}
\fi

\iffalse
\textbf{BEMERKUNGEN:\\
*Ich habe den Twist jetzt gelegt auf: Wir schauen uns LPN+Co an bzgl Speicher, aber auch was das in der Anwendung (Key Exchange) bedeutet\\[1ex]
*Es ware nicht schlecht, wenn wir noch ein paar Links setzen wuerden (Meyer $\to$  Memory, Eike $\to$ Reduktionen, letzteres waere fur die vermutlich auch nicht schlecht, siehe letztes Mal)\\[1ex]
}
\fi

\iffalse
\marc{Ich glaube, das alles da unten bis Summary kommt  jetzt extra}

\subsection{Topic}
Operations on encrypted data (streams), secure outsourcing of computations into the cloud, private information retrieval, secure computation of statistics on encrypted/authenticated/signed data, privacy techniques.

\subsection*{1.2 \hskip 0.3cm Thema}
Operationen auf verschl�sselten Daten(-str�men), sicheres Auslagern von Berechnungen in die Cloud, Private Information Retrieval, sichere Statistikberechnungen auf verschl�sselten/authentisierten/signierten Daten, Techniken zur Wahrung der Privatsph�re.

\subsection{Keywords}
Big Encrypted Data.

\subsection{Research area and field of work}
Theoretical computer science, complexity theory, cryptography, multi-party computation.

\subsection{Anticipated total duration}

The anticipated duration of the project is 3 years.

\subsection{Application period}

We apply for a period of 36 months funding, starting in June 2014.
\fi

\section*{Summary}
%\subsection{Summary}

The security of cryptographic protocols is related via reductions to the security of presumably hard problems. Such a reduction transforms any successful adversary against a protocol into one against the underlying problem such that, vice versa, the hardness of the problem implies the hardness of the protocol. In the past, cryptographers used classical number-theoretic problems like the RSA problem or the discrete logarithm problem. But these problems will become insecure once quantum computers become real. It is therefore no surprise that
the National Institute of Standards and Technology (NIST) announced a call for quantum-resistant cryptographic primitives in autumn 2016.

Most likely, the majority of reasonable candidates for the NIST competition will be based on hard problems from coding theory and lattices, such as Learning Parity with Noise (LPN) and Learning with Errors (LWE). NIST requests well-defined security levels of 128, 192 and 256 bits classically, and 64, 80, 128 bits quantumly. This means that for instantiating LPN with, say, 128 bit classical security, one has to make sure that any algorithm for LPN will need at least $2^{128}$ steps on a classical computer. 

Since running times of $2^{128}$ escape current computational power,
one nowadays typically studies medium security levels of $50-60$ bits in practical experiments, and then extrapolates via asymptotic formulas to the desired security level to make a reasonable security claim. However, besides running time, the memory requirement is a major factor. In fact, for the interesting problems LPN and LWE current algorithms also require to store an exponential number of samples. This huge amount of data and its storage requirement do not allow to run experiments even for medium security levels, and thus prevent a reliable prediction of cryptographically secure key sizes.

%\marc{das koennte noch 1-2 Saetze laenger sein:}
The current project studies low-memory algorithms for problems in coding and lattice theory, and how this translates to the case when these problems are deployed in one of the prevailing applications of modern cryptography, namely, key exchange protocols. 
%We therefore concentrate on extracting a maximum amount of information of cryptographic oracles  in a streaming-based manner.




\iffalse
The security of cryptographic protocols is related via reductions to the security of hard problems.
A cryptographic security proof states that any successful adversary that breaks a cryptographic protocol $\Pi$ in time $t$ can be translated into an algorithm that breaks a hard problem in time $t'$. In order to instantiate cryptographic protocols $\Pi$ efficiently, one has to first find a tight security proof that closely links $\Pi$ to a hard problem, i.e. a reduction for which $t'$ is not much larger than $t$. Second, one has to make sure that any algorithm for the hard problem runs in time larger than $t'$. 

In the past, cryptography relied on well studied hard problems from number-theory, such as factoring integers and computing discrete logarithms in elliptic curves. However, these problems stay no longer hard in the presence of quantum computers. Therefore, the NIST (National Institute of Standards and Technology) announced in autumn 2016 a call for Post-Quantum cryptographic primitives.
%, i.e. encryption, signature and key exchange. Candidates have to be submitted by Nov 2017 and then undergo a period of 3-5 of cryptanalytic research.

Most likely, the majority of reasonable candidates will be based on hard problems from coding theory and lattices, such as Learning Parity with Noise (LPN) and Learning with Errors (LWE). NIST requests well-defined security levels of 128, 192 and 256 bits classically, and 64, 80, 128 bits quantumly. This means that for instantiating LPN with e.g. 128 bit classical security, one has to make sure that any algorithm for LPN will need at least $2^{128}$ steps on a classical computer.

For making such a security claim in a reliable manner, one nowadays studies medium security levels of $50-60$ bits in practical experiments, and then extrapolates via asymptotic formulas to the desired security level. However, current algorithms for LPN and LWE for security level $b$ also require to store $2^b$ samples. This huge amount of data prevents to run experiments even for medium security levels, and thus prevents a reliable prediction of cryptographically secure key sizes.

The current project studies low-memory algorithms for problems in coding and lattice theory, and how this translates to the case when these problems are deployed in key exchange protocols. We therefore concentrate on extracting a maximum amount of information of cryptographic oracles  in a streaming-based manner.
\fi


%Thus, we require data structures for encrypted data. In addition, we have to make sure that there do not exist any built-in backdoors for cryptography that allow an illegitimate person, e.g. the NSA, to decrypt the information.

%Moreover, in several scenarios it is benefical to protect only certain parts of some encrypted data. E.g., for encrypted medical data, it is useful to compute certain statistics or correlations between data elements in order to study interactions of medicaments. But this should be possible without revealing the identities of the involved patients. 

%A big open and somewhat ignored question in cryptography is whether all these operations on encrypted data can be done in a streaming model, where the encrypted data is provided as a stream rather than a randomly accessible piece of data. 



%\newpage
\section*{Zusammenfassung}

Die Sicherheit kryptographischer Protokolle wird mittels Reduktionen auf die Sicherheit vermeintlich schwerer Probleme zur�ckgef�hrt. Solche Reduktionen transformieren einen erfolgreichen Angreifer auf ein Protokoll in einen Algorithmus f�r das zugrundeliegende Problem. Damit impliziert im Umkehrschluss die Schwere des Problems die Sicherheit des Protokolls. In der Vergangenheit nutzten Kryptographen klassische zahlentheoreti\-sche Probleme wie das RSA-Problem oder das Diskrete Logarithmus Problem. Diese Probleme sind allerdings leicht auf Quantenrechnern zu l�sen. Aufgrund der fortschreitenden Entwicklung von Quantenrechnern hat das National Institute of Standards and Technology (NIST) folgerichtig im Herbst 2016 einen Aufruf zur Einreichung quanten-resistenter kryptographischer Primitiven gestartet.

Sehr wahrscheinlich wird die Mehrzahl der praktikablen Kandidaten in diesem NIST-Wettbewerb auf der Schwere von Problemen der Kodierungs- und Gittertheorie basieren, wie z.B. Learning Parity with Noise (LPN) und Learning with Errors (LWE). NIST fordert wohldefinierte Sicherheitslevel von 128, 192 und 256 Bit auf klassischen Rechnern und 64, 80, 128 Bit auf Quantenrechnern. D.h. f�r eine Instantiierung von LPN mit z.B. 128 Bit klassischer Sicherheit muss sichergestellt werden, dass jeder LPN-Algorithmus mindestens $2^{128}$ Schritte auf einem klassischen Rechner ben�tigt.
 
Da $2^{128}$ Schritte weit jenseits derzeitiger Berechnungsm�glichkeiten liegen, studiert man heutzutage typischerweise mittlere Sicherheitslevel von $50-60$ Bit in praktischen Experimenten und extrapoliert mittels asymptotischer Formeln zum gew�nschten Sicherheitslevel, um vern�nftige Sicherheitsaussagen zu treffen. Allerdings spielt neben der Laufzeit der Speicherbedarf bei solchen Aussagen eine herausragende Rolle. F�r die interessanten Probleme LPN und LWE ben�tigen die heutzutage besten Algorithmen die Speicherung exponentiell vieler Samples. Diese riesigen Datenmengen und der damit verbundene Speicherbedarf machen derzeit eine Durchf�hrung von Experimenten selbst f�r mittlere Sicherheitslevel unm�glich und verhindern somit eine verl�ssliche Vorhersage von kryptographischen Schl�ssell�ngen.

Das folgende Projekt besch�ftigt sich mit Algorithmen mit geringem Speicherbedarf f�r Probleme der Kodierungs- und Gittertheorie und betrachtet den Einsatz dieser quanten-resistenten Probleme in einer der fundamentalsten Anwendungen moderner Kryptographie, dem Schl�sselaustausch-Protokoll.




%%%%%%%%%%%%%%%%%%%%%%%%%%%%%%%%%%%%%%%%%%%%%%%%%%%%%%%%%%%%%%%%%%%%%%%%%%%%%%%%%%%%%%

\iffalse
 McKinsey Global Institute
Big data: The next frontier for innovation, competition, and productivity
May 2011 
\fi

%\newpage
\section{State of the art and preliminary work}

Following the guidelines for applications we report on the state of the art as well as the PI's preliminary work. Although, technical speaking, this is \emph{not} a renewal proposal, the proposal here should continue our participation within the priority program SPP 1736~(Algorithms for Big Data). We therefore report about the achievements of our project SecOBig in the first phase of the program and the reason for discontinuing SecOBig in Section~\ref{sec:secobig}. 

\subsection{State of the art}

Big data processing in cryptography often refers to the efficiency of mounting attacks. Typically, if an adversary gets access to a sufficient amount of data and can provide enough resources, then the sought-after information becomes available. Remarkably, in most cases cryptographers focus on the time requirements of attacks in order to assess the security of solutions. Yet, the memory consumption for both storing large amounts of data and for executing attacks is an important factor, too.%
\footnote{A concrete example where this fact has been brought back to the center of attention is the (officially confirmed) initiative of the US National Security Agency (NSA) to build the so-called Utah Data Center for mass storage of data.}


\paragraph{The Impact of Memory Consumption.}
For analyzing and instantiating cryptographic systems one usually  focuses on the aspect of adversarial running time. For instance, for cryptographic reductions one usually does not care about memory requirements, such as for the storage for oracle queries and answers. The same is true for the study of the underlying hard problems. The reason is that, often, the fastest algorithm with running time $t$ also has a memory consumption of roughly $t$. 

In cryptography, however, the space requirement for an attack can be a significant factor.
%While it is in general a good idea to define cryptographic security in a conservative manner, this does not properly reflect practice. 
While performing $2^{60}$ operations today is considered to be feasible, even on a medium-sized computing cluster in a reasonable amount of time, any algorithm with RAM consumption $2^{60}$ bits will not be implementable in the near future. An Internet investigation shows that nowadays the largest supercomputers\footnote{e.g. the IBM 20-Petaflops cluster installed in Sequoia, Lawrence Livermore National Laboratory, California} have a RAM of at most 1.6 PB $<2^{54}$ bits. If an algorithm has to use external memory, then its running time usually slows down significantly.

Hence, for estimating the security of cryptographic constructions one should also consider an upper bound on the memory consumption. This in turn defines a need for finding efficient algorithms with small memory consumption. 

\paragraph{The Impact on LPN and LWE.}
In this proposal we combine the question of memory consumption with a recent development in cryptography, due to the potential advances in quantum computing. Nowadays the question pops up which cryptography can still be considered to be suitable to protect data, since classical problems like RSA or discrete logarithms will become insecure once quantum computers reach maturity.
%
%This is even more urgent for the most prominent candidates for Post-Quantum Cryptography. 

In November 2017, the National Institute of Standards and Technology (NIST) will open a call for candidates of cryptosystems for encryption, signature and key exchange, which are presumably immune to quantum attacks. The candidates will then undergo a period of 3-5 years of cryptanalytic research, before a recommendation is made. 
Most likely, a majority of these systems will be based on problems from coding and lattice theory, such as Learning Parity with Noise (LPN) and Learning with Errors (LWE). But the currently best algorithms for both problems have a memory consumption which is as large as their running time, making them useless for implemention even on medium size security levels. Let us have a closer look at LPN/LWE and their currently best algorithms. 


\paragraph{LPN and the BKW algorithm.} In LPN~\cite{DBLP:conf/focs/Alekhnovich03}, one has to find a secret $\mb s \in \F_2^n$ given access to an oracle that outputs samples of the form $(\mb a_i, b_i)$, where $\mb a_i \in_R \F_2^n$ and $b_i = \langle \mb a_i, \mb s \rangle + e_i$ for some Bernoulli error $e_i$. The algorithm of Blum, Kalai and Wasserman (BKW)~\cite{BKW} achieves the best currently known running time $t=2^{\bigO(\frac{n}{\log n})}$ for constant error, but suffers from requiring to store $t$ samples in memory. Due to its large sample complexity, there is no chance to speed up BKW quantumly, since LPN-samples are inherently classical.

\paragraph{LWE and Lattice Sieving.} LWE~\cite{Regev05} is a generalization of LPN to arbitrary fields $\F_q$. Namely, one has to find a secret $\mb s \in \F_q^n$ given access to an oracle that output samples of the form $(\mb a_i, b_i)$, where $\mb a_i \in_R \F_q^n$ and $b_i = \langle \mb a_i, \mb s \rangle + e_i$ for some discrete error $e_i$ whose distribution is centered around $0$ (e.g. a discrete Gaussian). Depending on the LWE-parameters, the best algorithm for LWE is either a generalization of BKW to $\F_q$~\cite{AlbrechtCFFP15} or lattice sieving~\cite{DBLP:conf/stoc/AggarwalDRS15}. Both algorithms achieve for cryptographic parameter settings a running time of $2^{\bigO(n)}$, consuming the same exponential amount of memory. Moreover, for both algorithms no significant speed-up via quantum search methods is known. 

\paragraph{Deployment of LPN and LWE in Key Exchange Protocols.}
Analogously to the problem of determining the necessary resources for mounting attacks, in order to make recommendations for secure choices for the underlying problems, we are interested in the security when these problems are deployed in more complex protocols.
%influence of the deployment of the problems like LWE and LPN. 
We are especially interested in securing communication data and the question how we can prevent attackers to decipher data which is stored now and potentially sifted through later, e.g., once quantum computers are available. The connection to the first part of the proposal is via the estimated hardness of potentially quantum-resistant problems such as LPN and LWE. 
%The research in that part will suggest parameter choices withstanding Big Data engineering efforts, such that we can build secure systems and recommend appropriate parameter choices. 

Securing communication data between two parties typically consists of composing a so-called key exchange protocol with a secure channel protocol.  With the secure key exchange protocol the participants establish a shared cryptographic key, which should be known only to them. Then, this key is used in a secure channel protocol to send the actual data in a confidential and authentic way. 

The focus for securing communication against advanced attacks, especially against quantum cryptanalysis, currently lies on the key exchange part. The  reason is that it is currently unclear if quantum attacks improve over classical attacks for the channel part. In contrast, most of the practically deployed key exchange protocols (such as the ones used in TLS~1.2 \cite{TLS12} and the future TLS~1.3 \cite{TLS13}) rely on number-theoretic problems which are highly amenable to quantum attacks. The best known examples are the paramount Diffie-Hellman based protocols which can be broken in polynomial time by quantum attackers.  


Concerning efforts to build quantum-resistant key exchange protocol, this area has gained quite some momentum, culminating in proposals to derive such protocols from (Ring-)LWE \cite{P14,BCNS15,ADPS16,BCDMNNRS16}. For the scheme in \cite{ADPS16}, the ``New Hope'' key exchange protocol, Google recently announced to experiment with the scheme in its Chrome browser \cite{Google}. Another recent proposal for a potentially quantum-resistant key exchange protocol is based on ideal lattices \cite{ZZDSD15}.

\paragraph{Shortcomings of current LPN-/LWE-based Key Exchange Protocols.}

Unfortunately, most of the quantum-immune key exchange candidates have at least one shortcoming. For example, the analysis of New Hope \cite{ADPS16} relies on the classical random oracle model, although it has been argued in \cite{BDFLSZ11} that quantum access to this idealized primitive should be preferred in such settings. Others, such as \cite{BCNS15,BCDMNNRS16} provide security in the ACCE notion of Jager et al.~\cite{JKSS12}, arguing security when the key exchange protocol is composed with an atomic channel. This at the moment excludes a modular analysis if instead a stream-based channel is used (as possible in TLS) \cite{FGMP15}. Moreover, all the analyses follow the classical choice of investigating primarily the run time and success probability of adversaries, mainly neglecting the memory consumption in the  reduction.


Furthermore, most of the aforementioned works  start with the unauthenticated key exchange setting, where parties and transmissions are not authenticated. They argue that this can be accomplished later by adding signatures. Here, however, it is unclear if breaking the authentication with a quantum computer afterwards could not lead to a break of current executions (since the secrets may still be in use later). Furthermore, in some cases this requires some changes to the structure, such that for example \cite{BCNS15,BCDMNNRS16} are, strictly speaking, not compatible to TLS.


Another shortcoming of the current proposals is that it they may not be easy to integrate into the upcoming TLS~1.3 protocol. While \cite{BCNS15,ADPS16,BCDMNNRS16} argue how this can be done for TLS~1.2 and provide impressive implementation and performance details, the next TLS~1.3 version adds significantly enhanced functional properties such as zero round-trip time steps. In the Diffie-Hellman setting this often necessitates to switch to other number-theoretic problems, such as the PRF-ODH assumption \cite{JKSS12}. It is unclear what this would mean for LWE- and LPN-based protocols, and if adopting a new assumption, for the memory consumption for this related problem.




\subsection{Preliminary Work of the PIs}

%!TEX root = memoc.tex

\paragraph{Memory-efficient LPN algorithm.} We have extensive expertise on the design of algorithms for decoding random binary linear codes~\cite{Codes11,Codes12,Codes15}. LPN  can be seen as a decoding problem in a random binary linear code generated by the LPN sample vectors $\mb a_i$, where the $b_i$ represent the erroneous codeword. %As opposed to classical decoding problems, in LPN one can freely choose the length of the code via the number of LPN samples.

We already have a preliminary paper that proposes new memory-efficient LPN algorithms~\cite{LPN}, both classically and also for the first time quantumly.  This work is based on our methods for decoding random binary linear codes, especially the May-Meurer-Thomae algorithm~\cite{Codes11}. The preliminary paper~\cite{LPN} is currently under submission, and is attached as supplementary material to this project proposal.

\paragraph{Asymptotic Complexity Analysis of LWE.} We also studied already the asymptotic complexity of existing algorithms for solving LWE concerning the metrics time, memory and number of samples. Our work~\cite{LWE} summarizes the state of the art and identifies the best algorithms for specific choices of the LWE parameters $n$, $q$ and the error distribution. However, all the algorithms in~\cite{LWE} suffer from their huge memory consumption, making them an inadequate choice for establishing secure LWE parameters in practice.
\paragraph{Key Exchange.}
We have extensively contributed to fundamental question about key exchange models \cite{BFWW11,FG14}, dealing with composability and multi-stage key exchange protocols. We have provided analyses of the (Diffie-Hellman based) TLS protocols \cite{BFSWW13,DFGS15}, both for versions 1.2 and the (candidates for) 1.3. We have also investigated the special modes and properties of the candidates for TLS version 1.3~\cite{DFGS15,FGSW16}.

\paragraph{Cryptographic Reductions.}
Since our research area is complexity-based cryptography, reductions are our main tool in conducting security proofs of cryptographic protocols and appear in the majority of our works. Concerning reductions themselves we have extensive expertise about notions of reductions \cite{F12,BBF13}. All the aforementioned works about key exchange involve reductions in the domain of key exchange.



\subsection{Preliminary Work in the first Project SecOBig}\label{sec:secobig}

In the first phase of the priority program both applicants have conducted a project called \emph{Security-Preserving Operations on Big Data} (\emph{SecOBig}).  The focus of this proposal here has changed now. In the following we report briefly on the achievements of \emph{SecOBig} (so far, with the projects still running for about 8 months) and the reason for the shift of topic and a fresh application (instead of a renewal application).

\paragraph{Achievements in SecOBig.}
SecOBig promised to work on efficient operations on secured data, both targeted as well as through the deployment of functional encryption and indistinguishable obfuscation, and certification of cryptographic primitives. At TU Darmstadt, Arno Mittelbach has been working on the project but has meanwhile left academia. \marc{RUB}
%
The following works have been published in the context of the project so far: \marc{RUB Sachen addieren}

\noindent
Peer-reviewed Publications:
\begin{itemize}
 \item   Marc Fischlin, Amir Herzberg, Hod Bin Noon, Haya Shulman:
Obfuscation Combiners. Crypto 2016. This work deals with the certification and obfuscation questions, as it shows how to build robust solutions in light of malicious obfuscators, and reports about implementation results.
%
\item 
Victoria Fehr, Marc Fischlin:
Sanitizable Signcryption: Sanitization over Encrypted Data. In submission, see also IACR cryptographic eprint archive 2015. Provides solutions to allow for controlled modifications of authenticated, encrypted data.
%
\item Rolf Egert, Marc Fischlin, David Gens, Sven Jacob, Matthias Senker, J\"orn Tillmanns: Privately Computing Set-Union and Set-Intersection Cardinality via Bloom Filters. ACISP 2015. This work deals with efficient operations on encrypted data for computing the number of shared elements in large sets. Topic has been inspired by the invited talk of Michael Goodrich about Bloom filters at the opening of the priority program.
%
\item Christina Brzuska, Pooya Farshim, Arno Mittelbach:
Random-Oracle Uninstantiability from Indistinguishability Obfuscation. TCC 2015. Relates to the question of obfuscation, and shows that the random oracle methodology for designing practical solutions may not be applicable in general.
%
\item Christina Brzuska, Arno Mittelbach:
Indistinguishability Obfuscation versus Multi-bit Point Obfuscation with Auxiliary Input. Asiacrypt 2014. Relates to the question of obfuscation and technical details about the realizability.
%
\item 
Christina Brzuska, Pooya Farshim, Arno Mittelbach:
Indistinguishability Obfuscation and UCEs: The Case of Computationally Unpredictable Sources, Crypto 2014. Relates to the question of obfuscation and shows that obfuscation can actually be used to show negative result.
%
\end{itemize}
%

\noindent
Theses:
\begin{itemize}
\item Sven Jacob. Realizing Cryptographic Protocols in the MapReduce-Framework. Ongoing Master Thesis, TU Darmstadt, 2016. Provides cryptographically secure protocols in the MapReduce framework, including implementations in Hadoop and Amazon's Elastic MapReduce (EMR) framework.
%
\item Arno Mittelbach. Random Oracles in the Standard Model --- A Systematic Study of Random Oracle (Un)Instantiability via Universal Computational Extractors and Obfuscation. Ph.D.~Thesis, TU Darmstadt, December 2015.
%
\item Kai Schwierczek. Approximation of the Maximum in Big Data, Master-Thesis TU Darmstadt, 2015. Touches the question how to compute the easy statistics (like the maximum or minimum) on encrypted outsourced data efficiently, by sacrificing precision.
%
\item Tobias Weber. Combiners for Robust Pseudorandom Number Generators, Master-Thesis TU Darmstadt, 2015. Deals with certification in the sense that one builds robust pseudorandom generators in the presence of some malicious generators.
%
\end{itemize}

\noindent
Others:
\begin{itemize}
\item Dagstuhl seminar about Public-Key Cryptography, organized by Fischlin, May, Rabin, and Pointcheval, September 2016. Big Data has been one of the topics in the seminar.
\end{itemize}


\paragraph{Shift of Topic.}
The main focus of the project \emph{SecOBig} was to perform operations on large amounts of cryptographically secured data. In the second phase, with project \emph{\memoc} we turn the focus to the research question what Big Data scenarios actually mean for cryptographic strengths. The reason for this transition this is twofold. First, it should allow a smoother collaboration within the priority program. Here, we especially refer to the projects \emph{Scalable Cryptography} of Hofheinz and Kiltz and \emph{Big-Data-DynAmO: Dynamic, Approximate, and Online Methods for Big Data} of Meyer. The former one touches related questions concerning tightness in cryptographic security proofs, and the latter one deals (among others) with memory resources. To best of our knowledge, both projects will be continued in the second phase of the priority program.

The other reason is based on the recent developments in the cryptographic community. One is that, in the past years, the has been a growing trend to perform general secure operation via so-called garbled circuits. This approach shows impressive performances, but neither one of the applicants here is an expert on this. Also, with NIST's recent call for post-quantum secure primitives and the growing interest by companies like Google, looking into this area in the context of Big data processing seems to be a more fashionable topic.





\subsection{Project-related publications}

%\marc{habe 4 von mir eignefuegt, dann kannst du noch 5 Stueck, glaube ich}
%\subsubsection*{Articles published by outlets with scientific quality assurance, book publications, and works accepted for publication but not yet published}

%laut DFG bei 2 Antragsstellern: 3 pro Foerderjahr, also nur 9 fuer uns.


\begingroup
\renewcommand{\section}[2]{}% eliminatinf reference section title / dirty hack


\begin{thebibliography}{99}

\bibitem{Reda}
Paul Baecher, Christina Brzuska, and Marc Fischlin. 
\emph{Notions of black-box reductions,
revisited}. In Advances in Cryptology - ASIACRYPT 2013 - 19th International
Conference on the Theory and Application of Cryptology and Information Security, Bengaluru, India, December 1-5, 2013, Proceedings, Part I, volume 8269 of Lecture Notes in Computer Science, pages 296-315. Springer, 2013.
%
%\bibitem{Codes12} Anja Becker, Antoine Joux, Alexander May, Alexander Meurer.
%	\textit{Decoding Random Binary Linear Codes in $2^{n/20}$: How 1+1=0 Improves Information Set Decoding} , In %Advances in Cryptology (Eurocrypt 2012), Lecture Notes in Computer Science, Springer-Verlag, 2012.
%
\bibitem{TLS13a}
Benjamin Dowling, Marc Fischlin, Felix G\"unther, and Douglas Stebila. 
\emph{A cryptographic analysis of the TLS 1.3 handshake protocol candidates}. In Proceedings of the 22nd ACM SIGSAC Conference on Computer and Communications Security, Denver, CO, USA, October 12-6, 2015, pages 1197-1210. ACM, 2015.	
%
\bibitem{LPN} Andre Esser, Robert K�bler, Alexander May.
\textit{LPN Decoded}, 2016, {\bf in submission}, attached to this proposal as complementary material.
%
\bibitem{Redb}
Marc Fischlin. \emph{Black-box reductions and separations in cryptography}. In Progress
in Cryptology - AFRICACRYPT 2012 - 5th International Conference on Cryptology
in Africa, Ifrance, Morocco, July 10-12, 2012. Proceedings, volume 7374 of Lecture
Notes in Computer Science, pages 413-422. Springer, 2012.
%
\bibitem{TLS13b}
Marc Fischlin, Felix G\"unther, Benedikt Schmidt, and Bogdan Warinschi. \emph{Key confirmation in key exchange: A formal treatment and implications for TLS 1.3}. In IEEE Symposium on Security and Privacy, SP 2016, San Jose, CA, USA, May 22-26,
2016, pages 452-469. IEEE Computer Society, 2016.
%
\bibitem{LWE}
Gottfried Herold, Elena Kirshanova, Alexander May.
\textit{On the Asymptotic Complexity of Solving {LWE}}, {IACR} Cryptology ePrint Archive, http://eprint.iacr.org/2015/1222, accepted for publication in Journal Design, Codes and Cryptography, 2015.
%
\bibitem{BDD} Elena Kirshanova, Alexander May, Friedrich Wiemer.
\textit{Parallel Implementation of BDD enumeration for LWE} , In International Conference on Applied Cryptography and Network Security (ACNS 2016), Lecture Notes in Computer Science, Springer-Verlag, 2016.
%
\bibitem{Codes11} Alexander May, Alexander Meurer, Enrico Thomae.
	\textit{Decoding Random Linear Codes in $O(2^{0.054n})$} , In Advances in Cryptology (Asiacrypt 2011), Lecture Notes in Computer Science, Springer-Verlag, 2011.
%
\bibitem{Codes15} Alexander May, Ilya Ozerov.
\textit{On Computing Nearest Neighbors with Applications to Decoding of Binary Linear Codes} , In Advances in Cryptology (Eurocrypt 2015), Lecture Notes in Computer Science, Springer-Verlag, 2015.
%
\end{thebibliography}
\endgroup

%brauchen wir nicht
%\subsubsection*{Other publications}


%\subsubsection{Patents}
%\paragraph{Pending}
%
%\paragraph{Issued}

%%%%%%%%%%%%%%%%%%%%%%%%%%%%%%%%%%%%%%%%%%%%%%%%%%%%%%%%%%%%%%%%%%%%%%%%%%%%%%%%%%%%%%


\section{Objectives and work programme}

\subsection{Anticipated total duration of the project}

The total duration time of the project is 36 months (3 years). The project duration spans from the actual project begin (presumably June 2017) to 36 months later. Funding of the DFG is requested for the entire duration of 36 months.


\subsection{Objectives}

The overall goal of the project is to advance the field of memory-efficient evaluations for the cryptographic problems LWE and LPN, in order to provide sound security estimates even if  required data exceeds reasonable memory bounds. These problems are primary candidates in NIST's search for future, quantum-secure primitives. The research should be carried out for the problems itself, but also for the deployment of the problems in the important setting of key exchange. To this end the following sub goals are:
\begin{description}
%
\item[Memory Efficient Combinatorial/Lattice-Based LWE Algorithms.] We will develop combinatorial and lattice-based algorithms for LWE, both classically and quantumly, with limited memory consumption. Our goal is to precisely predict cryptographic security levels as a function of the LWE parameters $n$, $q$ and the size of the error. Therefore, we will implement our algorithms, run them on medium size instances and extrapolate asymptotically to cryptographically relevant security levels. 
%
\item[Practical Cryptanalysis of NIST's Post-Quantum Candidates.] Having combinatorial and lattice-based algorithms for tackling LWE-based cryptographic constructions, we will study the security of recent (and upcoming) proposals for the NIST initiative, such as e.g. New Hope~\cite{ADPS16} and Frodo~\cite{BCDMNNRS16}.
%
 \item[Analyses of existing LWE/LPN-based Protocols:]
  The idea is to revisit the existing protocols in light of the findings about the memory requirements for LWE and LPN. The goal is to make statements about the security bounds based on these results. This requires to check the memory consumption of the reduction and to verify if solvers for the underlying problems can be directly used in the context of key exchange, e.g., if one can ``stream'' executions of the key exchange protocol to the solver.
  %
 \item[Combining TLS~1.3 with LWE/LPN:]
 The objective here is to make LWE- and LPN-based key exchange protocols resemble the TLS~1.3 structure more closely. We expect that this will require to introduce a ``PRF-ODH like'' assumption for these problems. The goals here are thus to (a) devise protocols, (b) give a reduction to a suitable LWE- or LPN-based problem, taking the memory requirements into account, and (c) to evaluate the security of the problem in light of the analyses of the original problems in the other part of the project.
 


\end{description}



\subsection{Work programme incl. proposed research methods}

The work programme is split into 4 parts.
Project parts A+B describe the work programme associated to Alexander May,
project parts C+D describe the work programme of Marc Fischlin. 
%Part C is a joint part.
The duration of the individual work packages and their dependencies are described in
Figure~\ref{fig:WPs}.

%\marc{hm, wir muessten da vielleicht noch die Abhaengigkeiten/Laufzeiten beschreiben. Ich wuerde sequentiell durchgehen: D1 (8 Mo), D2 (10 Mo), C1 (6 Mo), C2 (12 Mo), mit der Begruendung, dass ich in C erst auf eure ersten Analysen warten muss}


%!TEX root = memoc.tex
\newcounter{workpackage}
\renewcommand{\theworkpackage}{\Alph{workpackage}}
\setcounter{workpackage}{0}
\refstepcounter{workpackage}\label{am:wp:eins}

%Referenzen auf WP-Buchstaben am:wp:eins und am:wp:zwei, sowie mf:wp:eins,...

\paragraph{Work package \theworkpackage.1: Memory Efficient Combinatorial LWE Algorithms.}

Our starting point is our preliminary work on LPN~\cite{LPN} that we generalize to arbitrary fields $\F_q$. We expect to obtain a hybrid algorithm that for limited memory is close to decoding algorithms for LWE-type problems, and for large memory resembles BKW. Especially, analogous to the LPN case~\cite{LPN}, we will divide our algorithm in two steps for dimension-reduction and decoding. The first step makes use of some limited, available amount of memory to reduce the LWE dimension in a BKW-type manner as far as possible, whereas the second decode step solves a resulting LWE-samples in smaller dimension.

This will lead to an algorithm that can be optimally adapted to any given amount of memory. We will implement our algorithm in $C$, run experiments for medium-size LWE parameters and derive asymptotics that allow for extrapolation to large cryptographic instances.

\paragraph{Work package \theworkpackage.2: Memory Efficient Lattice Reduction.} 

Our starting point is the preliminary work~\cite{BDD}, where we implemented a two-step algorithm for LWE. In this algorithm, the first step reduces the lattice basis defined by the LWE-sample matrix. So as opposed to the algorithm in WP A.1, here we do not reduce the dimension but the size of the coefficients. The second step is then a parallel enumeration over all candidates for the LWE secret, which in lattice language is a solution to a closest vector promise problem.

Our work~\cite{BDD} suffers from the fact that we implemented some non-optimal, but memory-efficient, lattice reduction procedure. Recent lattice sieving techniques~\cite{DBLP:conf/stoc/AggarwalDRS15} are asymptotically much faster, but also require large memory consumption. We will explore possible tradeoffs in lowering the memory consumption of these algorithms by using techniques from streaming algorithms, similar to~\cite{LPN}. This will sacrifice a bit in running time, at the benefit of obtaining implementable and practical algorithms even for large lattice dimensions.

%--
\refstepcounter{workpackage}\label{am:wp:zwei}


\paragraph{Work package \theworkpackage.1: Memory Efficient Quantum Algorithms for LWE.}

With the invention of cryptographic systems for the era of quantum computing, it is mandatory to study the best quantum algorithms for LPN and LWE. Many classical algorithms can be significantly speeded up using Quantum Search Techniques, such as e.g. Grover search~\cite{Grover96}.

We will look at possible extensions and improvements of our algorithms in WP~\ref{am:wp:eins} by enhancing them with quantum techniques. This will settle cryptographic quantum key sizes for LWE, similar to the estimates that have been done in~\cite{LPN} for LPN.

In a quantum world, it seems to be even more comprehensible to focus on small memory consumption, since quantum computing devices currently suffer from scalability. Thus, we will focus on algorithms with a quantum memory that is limited linearly (or even sublinear) in the input size.

\paragraph{Work package \theworkpackage.2: Practical Cryptanalysis of NIST Proposals.} In WP~\ref{am:wp:eins} we develop algorithm for tackling LPN and LWE instances in practice. This will enable us to judge the security levels of current NIST post-quantum cryptographic proposals. Interesting candidates that we will analyze with our algorithms are e.g. New Hope~\cite{ADPS16} and Frodo~\cite{BCDMNNRS16}. 

As the deadline for submitting candidates to NIST is in Nov 2017, we expect to see many more interesting candidates based on the LPN and LWE within the next year. These candidates will by analyzed for security using our algorithms from WP~\ref{am:wp:eins}.



\paragraph{Workpackage \Alph{workpackage}.1: Memory-efficient Reductions}

\marc{to be done}

\paragraph{Workpackage \Alph{workpackage}.2}

\marc{to be done}

\stepcounter{workpackage}


\paragraph{Work package \Alph{workpackage}.1: Adaptation to TLS~1.2.}

In this work packages we adopt ideas from the TLS~1.2 concept \cite{TLS12} to the suggested LWE-based key exchange protocol designs. The main step is to investigate if one can remove the late signatures in \cite{BCNS15} which make the solution slightly TLS-non-conforming. As discussed in \cite{BCNS15} such a change would most likely imply to switch to a PRF-ODH like assumption, allowing the adversary to mount an active attack against the underlying LWE problem. As pointed out by Peikert \cite{P14}, the problem seems to be easy under such active attacks, though. 

Fortunately, not all is lost: For TLS~1.2 the analysis of Jager et al.~\cite{JKSS12} requires only a very limited form of active attacks in which the adversary can make a single chosen queries only. For such active attacks the LWE problem may still be hard. As an alternative, or second step, we consider other designs possibilities for the key exchange protocol, thwarting this problem by design.


\paragraph{Work package \Alph{workpackage}.2: Adaptation to TLS~1.3.}

This work packages looks into the possibility to adapt the ideas of previous LWE-based proposals to (the current draft of) TLS~1.3~\cite{TLS13}. Since TLS~1.3 will be fundamentally different from TLS~1.2 this at foremost requires us to check if the current solutions can be transferred at all. 

Next, we address the question if we can augment existing protocols by a 0RTT mode where one derives a fresh key without interaction by consulting previous communication data. As explained above, this presumably requires an even stronger PRF-ODH like assumption, where the adversary can make many active queries. Here the question which should be addressed is if one needs to make some restriction on the number of key exchange sessions in which material is re-used.



\paragraph{Work package \Alph{workpackage}.2: Analysis of Underlying Problems.}

In this work package we investigate the hardness of the underlying problem(s) proposed in Work packages \Alph{workpackage}.1 and \Alph{workpackage}.2. This investigation covers several aspects. First, we verify with the results of Work packages \marc{XXXX; link auf ein vorheriges WP} how hard the problem itself seems to be. Secondly, we try to relate the new problem (via memory-efficient reductions) to the standard LWE problem, or try to show that the problem is strictly stronger, by giving a black-box separation result.


\stepcounter{workpackage}





%+++ make sure it appears on page 2
\begin{figure}[ht]
\begin{center}
%\begin{landscape}
\begin{adjustbox}{max width=0.7 \textwidth}

 \begin{ganttchart}[vgrid, % display vertical grid
 title height=1, % make number-boxes higher
 bar/.style={fill=blue!80}, % reset style
 bar height=0.5, % ...
 group/.append style={fill=blue!40}, %  append to style
 group height=.3, % ...
 ]{1}{36} 
 
% \gantttitle{ \textbf{Workpackages \& Tasks - Overview} }{36} \\
 \gantttitlelist{1,...,36}{1} \\
 
  % Workpackage A (elem0, elem1, ...)
  \ganttgroup{Work package A}{1}{24} \\
  \ganttbar{Work package A.1}{1}{9} \\
  \ganttlinkedbar{Work package A.2}{10}{24} \ganttnewline
  
  % Workpackage B
  \ganttgroup{Work package B}{10}{36} \\
  \ganttbar{Work package B.1}{10}{12}
  \ganttbar{Work package B.1}{25}{27} \\
  \ganttbar{Work package B.2}{13}{36} \ganttnewline
    
  % Workpackage C
  \ganttgroup{Work package C}{19}{36} \\
  \ganttbar{Work package C.1}{19}{24} \\
  \ganttlinkedbar{Work package C.2}{25}{36} \ganttnewline
     
  % Workpackage D
  \ganttgroup{Work package D}{1}{18} \\
  \ganttbar{Work package D.1}{1}{8} \\
  \ganttlinkedbar{Work package D.2}{9}{18} \ganttnewline
   
  % Inter-Workpackage Links
    \ganttlink{elem2}{elem5}
    \ganttlink{elem1}{elem4}
    \ganttlink{elem1}{elem6}
    \ganttlink{elem2}{elem6}
    \ganttlink{elem1}{elem8}
    \ganttlink{elem2}{elem9}
 \end{ganttchart}
 
\end{adjustbox}
%\end{landscape}
%
\end{center}
\vskip -0.6cm
\caption{{\small Work package durations and dependencies.}}\label{fig:WPs}
\end{figure}


\subsection{Data handling}

The results of the theoretical work is planned to be published at conferences, workshops, and in journals. This ensures
dissemination and availability of the project's results. In addition, we will use the usual electronic archives
like IACR's \texttt{eprint} service, and the PI's home pages to make full versions available in a lasting way.
As for experimental data, like software, we will also make these public, as part of the corresponding publication,
and through electronic archives like GitHub.
%\marc{lizenz? Open Source?}


\subsection{Other information}

 %Please use this section for any additional information you feel is relevant which has not been provided elsewhere.
Not applicable.

\subsection{Descriptions of proposed investigations involving experiments on humans, human materials or animals}

Not applicable.

\subsection{Information on scientific and financial involvement of international cooperation partners}

Not applicable.


%%%%%%%%%%%%%%%%%%%%%%%%%%%%%%%%%%%%%%%%%%%%%%%%%%%%%%%%%%%%%%%%%%%

\section{Bibliography}


\begingroup
\renewcommand{\section}[2]{}% eliminatinf reference section title / dirty hack

%\begin{thebibliography}{ABCD}

%\item[] \hspace{-0.6cm} Own work is marked with *.

%\reversemarginpar
%\marginparwidth 1pt

%\def\shortbib{0}
\bibliographystyle{alpha}
\bibliography{lit,local_bib}


\iffalse
\bibitem[ABMRS13]{dedup2}
Martin Abadi, Dan Boneh, Ilya Mironov, Ananth Raghunathan and Gil Segev:
Message-Locked Encryption for Lock-Dependent Messages. CRYPTO 2013, LNCS, Springer, 2013.

\bibitem[ABCHSW12]{P2:ABCHSW12} J. H. Ahn, D. Boneh, J. Camenisch, S. Hohenberger, A. Shelat, and B. Waters. Computing
on Authenticated Data. In: TCC 2012. Ed. by R. Cramer. Vol. 7194. Lecture Notes in
Computer Science. Taormina, Sicily, Italy: Springer, Berlin, Germany, 2012, pp. 1-20.

\bibitem[BMS13]{P2:Backes} M. Backes, S. Meiser, and D. Schr\"oder. Highly Controlled, Fine-grained Delegation of Signing
Capabilities. IACR Cryptology ePrint Archive, 408/2013. 2013.

\bibitem[BGIRSV01]{DBLP:journals/eccc/ECCC-TR01-057}
B.~Barak, O.~Goldreich, R.~Impagliazzo, S.~Rudich, A.~Sahai, S.~P. Vadhan, and
  K.~Yang.
\newblock On the (im)possibility of obfuscating programs. 
\newblock CRYPTO 2001, LNCS, Springer. Full version: {\em Electronic Colloquium on Computational Complexity (ECCC)},
  8(057), 2001.

\bibitem[BBN07]{BBN07}
Mihir Bellare, Alexandra Boldyreva, Adam O'Neill: Deterministic and Efficiently Searchable Encryption. CRYPTO 2007, LNCS, Springer, 
pp.~535-552, 2007.

\bibitem[BKR13]{dedup}
Mihir Bellare, Sriram Keelveedhi, Thomas Ristenpart: Message-Locked Encryption and Secure Deduplication. EUROCRYPT, LNCS, Springer, 
pp.~296-312, 2013.

\bibitem[BN02]{P2AC:BelNev02} M. Bellare and G. Neven. Transitive Signatures Based on Factoring and RSA. In: Advances
in Cryptology - ASIACRYPT 2002. Ed. by Y. Zheng. Vol. 2501. Lecture Notes in
Computer Science. Queenstown, New Zealand: Springer, Berlin, Germany, 2002, pp. 397-414.



\bibitem[BPMO12]{BPMO12} Erik-Oliver Blass, Roberto Di Pietro, Refik Molva, Melek \"Onen: PRISM - Privacy-Preserving Search in MapReduce. Privacy Enhancing Technologies 2012, LNCS 7384, Springer, pp.~180-200, 2012.

\bibitem[BSN08]{BSN08} Alexandra Boldyreva, Serge Fehr, Adam O'Neill: On Notions of Security for Deterministic Encryption, and Efficient Constructions without Random Oracles. CRYPTO 2008, LNCS, Springer, pp.~335-359, 2008.

\bibitem[BT04]{online} 
Alexandra Boldyreva, Nut Taesombut: Online Encryption Schemes: New Security Notions and Constructions. CT-RSA 2004, LNCS, Springer, 
pp.~1-14, 2004.

\bibitem[BGLS03]{BGLS03}
D. Boneh, C. Gentry, H. Shacham, and B. Lynn:
Aggregate and Verifiably Encrypted Signatures from Bilinear Maps.
In proceedings of Eurocrypt 2003, LNCS 2656, Springer, pp. 416-432, 2003. 

\bibitem[BSW06]{DBLP:conf/eurocrypt/BonehSW06}
D.~Boneh, A.~Sahai, and B.~Waters.
\newblock Fully collusion resistant traitor tracing with short ciphertexts and
  private keys.
\newblock In S.~Vaudenay, editor, {\em EUROCRYPT}, volume 4004 of {\em Lecture
  Notes in Computer Science}, pages 573--592. Springer, 2006.

\bibitem[BSW11]{DBLP:conf/tcc/BonehSW11}
D.~Boneh, A.~Sahai, and B.~Waters.
\newblock Functional encryption: Definitions and challenges.
\newblock In Y.~Ishai, editor, {\em TCC}, volume 6597 of {\em Lecture Notes in
  Computer Science}, pages 253--273. Springer, 2011.

\bibitem[BF11]{P2EC:BonFre11} D. Boneh and D. M. Freeman. Homomorphic Signatures for Polynomial Functions. In:
Advances in Cryptology - EUROCRYPT 2011. Ed. by K. G. Paterson. Vol. 6632. Lecture
Notes in Computer Science. Tallinn, Estonia: Springer, Berlin, Germany, 2011, pp. 149-168.

\bibitem[BGI13]{P2:Boyle}
E. Boyle, S. Goldwasser, and I. Ivan. Functional Signatures and Pseudorandom Functions.
IACR Cryptology ePrint Archive, 401/2013. 2013.


\bibitem[CLX09]{P2RSA:ChaLimXu09}
E.-C. Chang, C. L. Lim, and J. Xu. Short Redactable Signatures Using Random Trees. In: CT-RSA 2009. Ed. by M. Fischlin. Vol. 5473. Lecture Notes in Computer Science.
San Francisco, CA, USA: Springer, Berlin, Germany, 2009, pp. 133-147.

\bibitem[CKLYBNO06]{ml} Cheng-Tao Chu, Sang Kyun Kim, Yi-An Lin, YuanYuan Yu, Gary R. Bradski, Andrew Y. Ng, Kunle Olukotun: Map-Reduce for Machine Learning on Multicore. NIPS, pp.~281-288, 2006.

\bibitem[DG04]{OSDI04} Jeffrey Dean and Sanjay Ghemawat:
 MapReduce: Simplified Data Processing on Large Clusters.
 OSDI'04: Sixth Symposium on Operating System Design and Implementation,
San Francisco, CA, December, 2004. 

\bibitem[D93]{P2Desmedt:1993:CSR:283751.283834} Y. Desmedt. Computer security by redefining what a computer is. In: Proceedings on the 1992-
1993 workshop on New security paradigms. NSPW 92-93. ACM, 1993, pp. 160-166.

\bibitem[EOM13]{EOM13} Kaoutar Elkhiyaoui, Melek \"Onen, and Refik Molva:
Privacy preserving delegated word-search in the cloud,
TCLOUDS 2013, Workshop on Trustworthy Clouds, in connection with ESORICS, 2013.

\bibitem[F12]{P2PKC:Freeman12} D. M. Freeman. Improved Security for Linearly Homomorphic Signatures: A Generic Framework.
In: PKC 2012: 15th International Workshop on Theory and Practice in Public Key Cryptography.
Ed. by M. Fischlin, J. Buchmann, and M. Manulis. Vol. 7293. Lecture Notes in Computer Science.
Darmstadt, Germany: Springer, Berlin, Germany, 2012, pp. 697-714.

\bibitem[CFW12]{P2PKC:CatFioWar12} D. Catalano, D. Fiore, and B. Warinschi. Efficient Network Coding Signatures in the Standard
Model. In: PKC 2012: 15th International Workshop on Theory and Practice in Public Key Cryptography.
Ed. by M. Fischlin, J. Buchmann, and M. Manulis. Vol. 7293. Lecture Notes in Computer
Science. Darmstadt, Germany: Springer, Berlin, Germany, 2012, pp. 680-696.

\bibitem[GGH13]{DBLP:conf/eurocrypt/GargGH13}
S.~Garg, C.~Gentry, and S.~Halevi.
\newblock Candidate multilinear maps from ideal lattices.
\newblock In T.~Johansson and P.~Q. Nguyen, editors, {\em EUROCRYPT}, volume
  7881 of {\em Lecture Notes in Computer Science}, pages 1--17. Springer, 2013.

\bibitem[GGHRSW13]{DBLP:journals/iacr/Garg13}
S.~Garg, C.~Gentry, S.~Halevi, M.~Raykova, A.~Sahai, and B.~Waters.
\newblock Attribute-based encryption for circuits from multilinear maps.
\newblock {\em FOCS}, 2013.

\bibitem[GKPVZ13a]{DBLP:conf/crypto/GoldwasserKPVZ13}
S.~Goldwasser, Y.~T. Kalai, R.~A. Popa, V.~Vaikuntanathan, and N.~Zeldovich.
\newblock How to run turing machines on encrypted data.
\newblock In R.~Canetti and J.~A. Garay, editors, {\em CRYPTO (2)}, volume 8043
  of {\em Lecture Notes in Computer Science}, pages 536--553. Springer, 2013.

\bibitem[GKPVZ13b]{DBLP:conf/stoc/GoldwasserKPVZ13}
S.~Goldwasser, Y.~T. Kalai, R.~A. Popa, V.~Vaikuntanathan, and N.~Zeldovich.
\newblock Reusable garbled circuits and succinct functional encryption.
\newblock In D.~Boneh, T.~Roughgarden, and J.~Feigenbaum, editors, {\em STOC},
  pages 555--564. ACM, 2013.



%\bibitem{AG} Sanjeev Arora, Rong Ge, ``Learning Parities with Structured Noise'', Electronic Colloquium on Computational Complexity (ECCC) 17: 66 (2010)

\bibitem[JMSW02]{P2RSA:JMXW02}
R. Johnson, D. Molnar, D. X. Song, and D. Wagner. Homomorphic Signature Schemes. In: Topics
in Cryptology - CT-RSA 2002. Ed. by B. Preneel. Vol. 2271. Lecture Notes in Computer Science.
San Jose, CA, USA: Springer, Berlin, Germany, 2002, pp. 244-262.

\bibitem[KR11]{KR11} Seny Kamara, Mariana Raykova: Parallel Homomorphic Encryption. IACR Cryptology ePrint Archive 2011, No.~596, 2011.
See also WAHC'13 - Workshop on Applied Homomorphic Cryptography, Associated with Financial Crypto \& Data Security 2013.

\bibitem[KSV10]{KSV10} H. Karloff, S. Suri, and S. Vassilvitskii. A model of computation for mapreduce. In Symposium on
Discrete Algorithms (SODA '10), pages 938--948. SIAM, 2010.

\bibitem[LJLC12]{LJLC12}
Jingwei Li, Chunfu Jia, Jin Li, Xiaofeng Chen: Outsourcing Encryption of Attribute-Based Encryption with MapReduce. ICICS 2012,
LNCS 7618, Springer, pp.~191-201, 2012.

\bibitem[LD10]{LD10} J. Lin and C. Dyer. Data-Intensive Text Processing with MapReduce. Morgan \& Claypool, 2010.


\bibitem[LMRS04]{LMRS04} 
Anna Lysyanskaya,
Silvio Micali,
Leonid Reyzin,
Hovav Shacham.
Sequential Aggregate Signatures from Trapdoor Permutations.
In EUROCRYPT 2004, LNCS 3027, Springer, pp.~74-90, 2004.


\bibitem[MR02]{P2RSA:MicRiv02b}
S. Micali and R. L. Rivest. Transitive Signature Schemes. In: CT-RSA 2002.
Ed. by B. Preneel. Vol. 2271. Lecture Notes in Computer Science. San Jose, CA, USA: Springer,
Berlin, Germany, 2002, pp. 236-243.


\bibitem[PHGR12]{DBLP:conf/sp/ParnoHG013}
B.~Parno, J.~Howell, C.~Gentry, and M.~Raykova.
\newblock Pinocchio: Nearly practical verifiable computation.
\newblock In {\em IEEE Symposium on Security and Privacy}, pages 238--252. IEEE
  Computer Society, 2013.


\bibitem[SW05]{DBLP:conf/eurocrypt/SahaiW05}
A.~Sahai and B.~Waters.
\newblock Fuzzy identity-based encryption.
\newblock In R.~Cramer, editor, {\em EUROCRYPT}, volume 3494 of {\em Lecture
  Notes in Computer Science}, pages 457--473. Springer, 2005.


\bibitem[T10]{T10} Javier Tordable: MapReduce for Integer Factorization. CoRR abs/1001.0421, 2010.


%[13] M. T. Goodrich and M. Mitzenmacher. Mapreduce parallel
%cuckoo hashing and oblivious ram simulation. CoRR,
%abs/1007.1259, 2010.

\fi

%\end{thebibliography}

\endgroup


%%%%%%%%%%%%%%%%%%%%%%%%%%%%%%%%%%%%%%%%%%%%%%%%%%%%%%%%%%%%%%%%%%%


\section{Requested modules/funds}
%Explain each item for each applicant (stating last name, first name).

\subsection{Basic Module}

\subsubsection{Funding for Staff}

We apply for the following funding for staff, following the DFG's personal rates for 2016 (DFG-Vordruck 60.12). 
Due to competitive nature of positions in Computer Science and in the area of IT security, we request
funding for full doctoral student positions (100\%). We apply for funding of two doctoral
students for the entire duration of the project. One student will be associated to TU Darmstadt (TUD)
and work on packages \ref{mf:wp:eins} and \ref{mf:wp:zwei}, the other student will be associated to Ruhr-University Bochum (RUB)
and work on packages \ref{am:wp:eins} and \ref{am:wp:zwei}. Student assistents (one for each project partner) have been calculated with 10h/months
for the full duration of the project, with average costs of 11 EUR/h. They should support the implementations
in the corresponding work packages.

\quad % make more space

\noindent {\footnotesize
\begin{tabular}{|l||l|l|l||r|r|r|}
 \hline No &  Type & assoc.~to & Description & Y1 & Y2 & Y3\\
 \hline \hline
 1 & Doctoral Student, 100\%  & TUD & WP \ref{mf:wp:eins} and \ref{mf:wp:zwei} & 61,800\euro & 61,800\euro & 61,800\euro \\
 2 & Doctoral Student, 100\%  & RUB& WP \ref{am:wp:eins} and \ref{am:wp:zwei} & 61,800\euro & 61,800\euro & 61,800\euro\\
 3 & Student Assistent, 100\% & TUD & Implementations & 1,320\euro & 1,320\euro & 1,320\euro\\
 4 & Student Assistent, 100\% & RUB & Implementations & 1,320\euro & 1,320\euro & 1,320\euro \\
 \hline
  & Total Amount  & & & 126,240\euro & 126,240\euro & 126,240\euro\\ \hline 
\end{tabular}
}

\quad

%The doctoral students should ...

%{\em two TVL 13-position} and 
%{\em two studentische Hilfskraft position} (10 h per week).
%We apply for a 3-year funding period of all positions, starting in June 2014.\\


\subsubsection{Direct Project Costs}


%\paragraph{Equipment up to EUR 10,000, Software and Consumables.\\}
%\noindent Not requested.
%\marc{hier brauchen wir diesmal nix, oder?}
%We request an overall funding of 3,000 EUR for each applicant over the entire project duration, 6,000 EUR in total,
%to carry out experiments in Amazon's commercial Elastic MapReduce. Using Amazon's platform instead of local
%computing centers of both universities allows for an easier benchmarking and a smoother collaboration.

\paragraph{Travel Expenses.}
We apply for 8,000 Euro travel funding per year, i.e. for a total travel funding of 24,000 Euro.
This funding will cover expenses of the project partners for traveling between Darmstadt and Bochum
(500 EUR per year per partner), as well as attendances of potential  meetings of the SPP (500 EUR per year per 
partner). In addition, we calculate with 3000 EUR per year and partner for presenting the project results
at internationally renowned conferences and workshops (where we assume 1,500 EUR for an intercontinental conference, 
1,000 EUR for a European conference, and 500 EUR for a workshop attendance).


\iffalse
\paragraph{Visiting Researchers (excluding Mercator Fellows)}

\paragraph{Expenses for Laboratory Animals}


\paragraph{Other Costs}

\paragraph{Project-related publication expenses}


\subsection{Instrumentation}

\paragraph{Equipment exceeding Euro 10,000}


\paragraph{Major Instrumentation exceeding Euro 50,000}


\subsection{Module Temporary Position for Funding}


\subsection{Module Replacement Funding}
\fi

\iffalse %nicht relevant
4.4	Module Temporary Clinician Substitute


[Text]
4.5	Module Mercator Fellows

[Text]

4.6	Module Workshop Funding

[Text]

4.7	Module Public Relations Funding

[Text]

\fi

%%%%%%%%%%%%%%%%%%%%%%%%%%%%%%%%%%%%%%%%%%%%

\section{Project requirements}

\subsection{Employment status information}

\noindent
{\bf Marc Fischlin}, Dr.rer. nat \\
W3-Professor\\
%Postdoc TU Darmstadt (tenured, on sabbatical for Heisenberg Professorship)\\
Year of birth: 1973, Nationality: German\\[0.3cm]
Cryptography and Complexity Theory\\
Department of Computer Science\\
Technische Universit\"at Darmstadt\\
Karolinenplatz 5, 64289 Darmstadt, Germany\\[0.3ex]
Phone office: +49-(0)6151/16-25730\\
Fax office: +49-(0)6151-16-22487\\
E-Mail: marc.fischlin@cryptoplexity.de \\[0.3cm]
Privat address: Charlotte-Posenenske-Str.58, 65197 Wiesbaden\\
Phone private: +49-(0)611/9882730\\ \\

\noindent {\bf Alexander May}, Dr. rer. nat\\
W3-Professor \\
Year of birth: 1974, Nationality German \\[0.3cm]
Faculty of Mathematics, NA 5/73\\
Ruhr-University Bochum\\
Universit\"atsstr. 150, 44801 Bochum\\[0.3cm]
Phone office: 0234/32-23261\\
Fax office: 0234/32-14430 \\
E-Mail: alex.may@rub.de \\[0.3cm]
Privat address: Im Haarmannsbusch 34, 44797 Bochum\\
Phone private: 0151-64967032\\
%

% For each applicant, state the last name, first name, and employment status (including duration of contract and funding body, if on a fixed-term contract).

%[Text]  

\subsection{First-time proposal data}
%Only if applicable: Last name, first name of first-time applicant.
\vspace*{-1.5ex}
Not applicable.
%[Text]

\subsection{Composition of the project group}

%List only those individuals who will work on the project but will not be paid out of the project funds. State each person’s name, academic title, employment status, and type of funding.
\textbf{TU Darmstadt:}
\begin{itemize}
\item Prof. Dr. Marc Fischlin, Chair for Cryptography and Complexity Theory. Funded through TU Darmstadt. %(funded through Heisenberg-Program FI 940/3-1 of DFG) 
%\item Dr. Pooya Farshim (funded through Heisenberg grant FI 940-1/1)
%\item Dipl.-Inform. Paul Baecher (funded through Heisenberg grant FI 940-1/1)
\item M.Sc.(Math) Jacqueline Brendel. Funded through DFG Doctoral College 2050 Privacy and Trust for Mobile Users.
\item M.Sc.(Math) Victoria Fehr. Funded through TU Darmstadt; expected to leave till project start.
\item M.Sc.(Math) Tommaso Gagliardoni. Funded through BMBF/Hesse Security Competence Center CRISP; expected to leave till project start.
\item M.Sc.(CS), M.Sc.(CS) Felix G\"unther. Funded through DFG Collaborative Research Center 1119 CROSSING.
\item M.Sc.(Math) Christian Janson. Funded through TU Darmstadt. 
\item M.Sc.(Math) Giorgia Azzurra Marson. Funded through DFG Collaborative Research Center 1119 CROSSING; expected to leave till project start.
\item M.Sc.(CS) Sogol Mazaheri. Funded through TU Darmstadt.
%\item M.Sc.(CS) Arno Mittelbach (funded through State Hesse, LOEWE center CASED) 
\end{itemize}
Support: 1 secretary.

\quad

\noindent
{\bf Ruhr-University Bochum:}
\begin{itemize}
\item Prof. Dr. Alexander May, Chair for Cryptology \& IT security. Founded through RU Bochum.
\item Dipl.-Math. Elena Kirshanova. Funded through RU Bochum; expected to leave till project start.
\item Dipl.-Math. Robert K�bler. Funded through DFG-SPP 1736 -- Big Data.
\item MSc (Math) Leif Both. Funded through third party.
\item Dipl.-Ing. Matthias Minihold. Funded through ECRYPT-CSA within EU H2020. 
\item Dip.-Ing. Andre Esser. Founded through DFG-GRK 1817 -- Ubiquitous Cryptography
\end{itemize}
Support: 1 secretary and 1 technical assistant.



\subsection{Cooperation with other researchers}

\subsubsection*{Researchers with whom you have agreed to cooperate on this project}

%We plan to continue our cooperations with the following colleagues:

%\marc{ziemlich dumm eigentlich, dass nach uns aufzuteilen; macht keinen kollaborativen Eindruck...}


We plan to cooperate with Prof.~Dennis Hofheinz (KIT, Germany) and Prof.~Eike Kiltz (RUB, Germany) on the question of memory-efficient reductions. Prof.~Hofheinz and Prof.~Kiltz have participated in the first phase of the Priority Program 1736 (Algorithms for Big Data) jointly on a project about tight security reductions and plan to continue their collaboration in the next phase.

We also plan to cooperate with Prof.~Ulrich Meyer (Johann Wolfgang Goethe-Universit�t Frankfurt, Germany) on the question whether external memory techniques can be successfully applied for the LPN/LWE scenario. Prof.~Meyer had a project on online methods in the first phase and plans to continue this project in the next phase.





%\marc{Meyer}


\iffalse
\noindent
\textbf{Marc Fischlin:}
%
\begin{itemize}
\item Prof.~Eike Kiltz (RUB, Germany)
%\item Prof.~Douglas Stebila (McMaster University, Canada)
%\item Prof.~Bogdan Warinschi (U Bristol, UK)
\end{itemize}

\noindent
\textbf{Alexander May:}
%
\begin{itemize}
\item Prof. Dr. Antoine Joux (University Versailles), G\"odel prize 2013, visis our group in Nov 2013 for a research stay
\item Prof. Dr. Johannes Bl\"omer (Paderborn University)
\item Prof. Dr. Christian Sohler (TU Dortmund), joint seminar ```Perlen der Theoretischen Informatik''' since 2009
\item Prof. Dr. Hans Simon (Ruhr-University Bochum)
\end{itemize}
%
\fi

\newpage
\subsubsection*{Researchers with whom you have collaborated scientifically within the past three years}

\textbf{Marc Fischlin:}
\begin{itemize}
\item Prof.~Michael Backes (U Saarland, Germany)
\item Dr.~David Bernhard, Prof.~Bogdan Warinschi (U Bristol, UK)
\item Dr.~Jean Paul Degabriele, Prof.~Kenny Paterson (RHUL, UK)
\item Prof.~Amir Herzberg (Tel-Aviv University)
\item Dr.~Anja Lehmann (IBM Zurich, Switzerland)
\item Prof.~Krzysztof Pietrzak (IST, Austria)
\item Dr.~Benedikt Schmidt (IMDEA, Madrid, Spain)
\item Prof.~Dominique Schr\"oder (U N\"urnberg-Erlangen, Germany)
\end{itemize}

\noindent
\textbf{Alexander May:}
\begin{itemize}
%\item Prof. Eike Kiltz (Ruhr-University Bochum)
\item Prof. Johannes Bl\"omer (Paderborn University)
\item Prof. Christian Sohler (TU Dortmund)
\item Prof. Alon Rosen (Herzliya, Israel)
\end{itemize}


\subsection{Scientific equipment}

Not requested.
%Besides standard equipment the project partners 
%\textbf{ich brauche nix besonderes!!!!}


\subsection{Project-relevant interests in commercial enterprises}

Not applicable.

%%%%%%%%%%%%%%%%%%%%%%%%%%%%%%%%%%%%%%%%%%
\section{Additional information}

%If applicable, please list proposals requesting major instrumentation and/or those previously submitted to a third party here.


We have not requested funding for this project from any other sources. In the event that we submit such a request, we will inform the Deutsche Forschungsgemeinschaft immediately.

The DFG liaison officer's of TU Darmstadt and Ruhr-University Bochum will be informed of this research funding request.



%\newpage

\iffalse
%\newpage
\section{Prerequisites for carrying out the project}

For the project we do not have other funding sources. TU Darmstadt and Ruhr-University Bochum will provide all other required facilities (office, computing resources, etc.)

\subsection{Our team -- Organisational structure of our groups}
%
{\bf Ruhr-University Bochum:}
\begin{itemize}
\item Prof. Dr. Alexander May, chair of the group Crypto \& IT security
\item Emmy Noether group leader Dr. Christopher Wolf (associated)
\item Dipl.-Math. Gottfried Herold 
\item Dipl.-Ing. Ilya Ozerov (currently financed by SPP 1307 -- Algorithm Engineering) 
\item Dipl.-Math. Elena Kirshanova (currently financed by GRK 1817 -- Ubiquitous Cryptography)
\item Dipl.-Ing. Stefan Hoffman (third party funding) 
\end{itemize}

Support: 1 secretary and 1 technical assistant

\subsection{Cooperation with other scientists}

We plan to continue our cooperations with the following colleagues.
%
\begin{itemize}
\item Prof. Dr. Antoine Joux (University Versailles), G�del prize 2013, visis our group in Nov 2013 for a research stay
\item Prof. Dr. Johannes Bl�mer (Paderborn University)
\item Prof. Dr. Christian Sohler (TU Dortmund), joint seminar ```Perlen der Theoretischen Informatik''' since 2009
\item Prof. Dr. Hans Simon (Ruhr-University Bochum)
\end{itemize}
%

\subsection{Scientific equipment}

fully existing

\subsection{Running costs for material}

not necessary

\subsection{Other requirements}

none


%%%%%%%%%%%%%%%%%%%%%%%%%%%%%%%%%%%%%%%%%%%%%%%%%%%%%%%%%%%%%%%%%%%%%%

%\newpage
\section{Declarations}


We have not requested funding for this project from any other sources. In the event that we
submit such a request, we will inform the Deutsche Forschungsgemeinschaft immediately.

The DFG liaison officer's of TU Darmstadt's and Ruhr-University Bochum will be informed of this research funding request.


%%%%%%%%%%%%%%%%%%%%%%%%%%%%%%%%%%%%%%%%%%%%%%%%%%%%%%%%%%%%%%%%%%%%%%

\section{Signature}

\vspace{2cm}

TU Darmstadt, 30.09.2013 \hspace{4cm} (Marc Fischlin) \hspace{4cm} Bochum, 15.03.2011 \hspace{4cm} (Alexander May) \\


%%%%%%%%%%%%%%%%%%%%%%%%%%%%%%%%%%%%%%%%%%%%%%%%%%%%%%%%%%%%%%%%%%%%%%

\section{Attachments}

\begin{itemize}
\item CV of the applicants
\item Paper: A. May, I. Ozerov, ``On Merging Lists Consistently: Solving Subset Sum in $2^{0.287n}$'', currently in submission.
\end{itemize}

%%%%%%%%%%%%%%%%%%%%%%%%%%%%%%%%%%%%%%%%%%%%%%%%%%%%%%%%%%%%%%%%%%%%%%

\newpage
\subsection{Prof. Dr. rer. nat. Alexander May}

born on March 22, 1974 in Friedberg (Germany)\\
Ruhr-University Bochum\\
Chair for Cryptology and IT-Security\\
Faculty of Mathematics\\
Phone: +49 234 32 23261\\
Email: alex.may@rub.de\\

\subsubsection*{Education \& Professional Experience}
\begin{tabular}{ll}
{10.93 -- 07.99} & Computer Science study, J.W. Goethe-Universit\"at in Frankfurt/Main \\
{09.99 -- 02.00} & PhD, Computer Science Department, ETH Z\"urich \\
{03.00 -- 12.03} & PhD, Computer Science Department, Paderborn University \\
{01.04 -- 09.05} & Post-Doc in DFG-Priority Programme \\
%
& {``Sicherheit in der Informations- und
Kommunikationstechnik''}\\
%
{10.05 -- 09.07} & Juniorprofessor, Cryptographic Protocols, TU Darmstadt \\
{since 10.07} & W3-Professor, Cryptology and IT-Security, Ruhr-University Bochum \\
\end{tabular}




\vskip 0.3cm



%\subsubsection*{Academic Honors and Awards}

%\begin{tabular}{ll}
%
%2009 & Best Paper Award PKC together with Maike Ritzenhofen for the paper \\
%& \textit{``Implicit Factoring: On Polynomial Time Factoring Given Only an Implicit Hint''} \\
%2006 & Best Paper Award PKC together with Daniel Bleichenbacher for the paper \\
%&  {\it ``New Attacks on RSA with Small Secret CRT-Exponents''} \\
%2005 & Price of Faculty Electrical Engineering, Computer Science and Mathematics\\
%& for the PhD thesis  {\it ``New RSA Vulnerabilities Using Lattice Reduction Methods''} \\
%& \\
%2007 & Price of Student Body Computer Science at TU Darmstadt for the best lecture\\
%2006 & Price of Student Body Computer Science at TU Darmstadt for the best lecture\\
%2004 & Weierstra\ss-Price of Faculty Electrical Engineering, Computer Science and Math.\\
%&  at Paderborn University for excellent teaching\\
%\end{tabular}


\subsubsection*{Research Grants Relevant for the Application}

\begin{itemize}
\item DFG-GRK 1817 Ubiquitous Cryptography, ``Outsourced Computation'', since 2013
\item ERC Starting Grant 307952 (Partner), ``Fast and Sound Cryptography'', since 2012 
\item DFG-SPP 1307 Algorithm Engineering, ``Algorithms for Subset Sum, Lattices and Linear Codes'', 2011-13
\item DFG RUB-RS (ExIni), ``Coding-Based Cryptanalysis'', 2009-12
\item DFG, ``Lattice-Based Solving of Polynomial Equations'', 2007-10
\item EU, ECRYPT II -- European Network of Excellence in Cryptology, since 2008
\item DFG-SPP 1079 IT Security, ``Lattice Attacks on RSA'', 2003-2005
\end{itemize}

\subsection*{Brief Description of the Chair for Cryptology and IT-Security}
The research group interests cover all aspects of modern cryptology, algorithmic number theory and discrete mathematics.
The chair's special focus lies on algorithmic aspects of cryptography.
Our spectrum of cryptographic applications covers problems from factoring, discrete logarithms, lattice and coding theory, diophantine equations and quantum algorithms.


\vskip 0.3cm


\subsubsection*{Five Publications most Relevant for the Application}
%\footnote{Die wichtigsten Beitr\"age sind \textbf{fett} gedruckt.}}

\vspace{0.0cm}


%\textbf{Book chapter  (reviewed)}

%\begin{itemize}
%\item \textit{``Using LLL-Reduction for Solving RSA and Factorization Problems: A Survey''} \\
%In ``The LLL Algorithm --Survey and Applications'', Editors Phong Nguyen and Brigitte Vall\'ee, Springer, 2010.
%\end{itemize}



%\noindent \textbf{Journals (reviewed)}

%\begin{itemize}
%\item \textit{``Deterministic Polynomial Time Equivalence of Computing
%the RSA Secret Key and Factoring''}  with Jean-S\'ebastien Coron \\
%Journal of Cryptology, 2007.

%\item \textit{``Perspectives for Cryptographic Long-Term Security''} with Johannes Buchmann and Ulrich Vollmer \\
%Communications of the ACM, Vol. 94(9), 2006.

%\item \textit{``Krypto 2020''} with Johannes Buchmann, Erik Dahmen and Ulrich Vollmer \\
%KES -- The Information Security Journal, Vol. 5, 2005.

%\end{itemize}

%\noindent \textbf{Conference Papers (reviewed)}

%\vspace{0.8cm}

\begin{itemize}
\item Saqib Kakvi, Eike Kiltz, Alexander May, \textit{``Certifying RSA''},  In Advances in Cryptology (Asiacrypt 2012), Lecture Notes in Computer Science, Springer-Verlag, 2012.
\item Anja Becker, Antoine Joux, Alexander May, Alexander Meurer, 
	\textit{``Decoding Random Binary Linear Codes in $2^{n/20}$: How 1+1=0 Improves Information Set Decoding''} , In Advances in Cryptology (Eurocrypt 2012), Lecture Notes in Computer Science, Springer-Verlag, 2012.
\item Alexander May, Alexander Meurer, Enrico Thomae
	\textit{``Decoding Random Linear Codes in $O(2^{0.054n})$''} , In Advances in Cryptology (Asiacrypt 2011), Lecture Notes in Computer Science, Springer-Verlag, 2011.
\item Jake Loftus, Alexander May, Nigel P. Smart, Frederik Vercauteren
	\textit{``On CCA-Secure Somewhat Homomorphic Encryption''} , In Selected Areas in Cryptography (SAC 2011), Lecture Notes in Computer Science, Springer-Verlag, 2011.
\item Wilko Henecka, Alexander May, Alexander Meurer
	\textit{``Correcting Errors in RSA Private Keys''}, In Advances in Cryptology (Crypto 2010), Lecture Notes in Computer Science, Springer-Verlag, 2010.
\end{itemize}

\fi

\end{document}


\iffalse DFG Vorlage

Project Description - Project Proposals


[First name  last name, city of all applicants] 
 





2	Objectives and work programme
2.1	Anticipated total duration of the project

[Text]
2.2	Objectives

[Text]
2.3	Work programme incl. proposed research methods
For each applicant

[Text]
2.4	Data handling

[Text]

2.5	Other information
Please use this section for any additional information you feel is relevant which has not been provided elsewhere.

[Text]

2.6	Descriptions of proposed investigations involving experiments on humans, human materials or animals

[Text]

2.7	Information on scientific and financial involvement of international cooperation partners

[Text]



3	Bibliography

[Text]




\fi


\iffalse
%===============
I Priority Programme
For individual project proposals within an established Priority Programme, note that the fund-ing duration (part A of the proposal) and the funding periods are specified in the call for pro-posals.
For the project description (part B of the proposal) note the following:
Each proposal must be accompanied by a description of how the project is integral to the Priority Programme, both in terms of subject matter and organisation. This includes a de-scription of the cooperation with others participating within the Priority Programme. The en-visaged realisation of the project in cooperation with other applicants may be demonstrated in particular by the joint training of early career researchers, or the use of methods by multi-ple projects as part of a network.
DFG form 54.01 - 04/13 page 11 of 14
Deutsche Forschungsgemeinschaft
Kennedyallee 40 ∙ 53175 Bonn ∙ postal address: 53170 Bonn
phone: + 49 228 885-1 ∙ fax: + 49 228 885-2777 ∙ postmaster@dfg.de ∙ www.dfg.de DFG
All applicants involved in submitting a proposal within an established Priority Programme are obliged to promptly provide the overall coordinator with all of the information necessary for drawing up the interim reports and the final report for the Priority Programme.
\fi