\documentclass[12pt]{article}
\usepackage[a4paper,hmargin=1in,vmargin=1.25in]{geometry}

%\usepackage{german}
\usepackage{latexsym,wrapfig}
\usepackage{amsmath,amssymb}
\usepackage{epsfig,color,colordvi}
\usepackage{graphics}
\usepackage{theorem}
\usepackage{pstricks}
\usepackage{amsfonts}
\usepackage[T1]{fontenc}
\usepackage{eurosym}

%\textwidth 16.1cm      %changea4.tex
%\textheight 170mm      %
%\textheight 223mm      %
%\evensidemargin 0mm    %
%\oddsidemargin 0mm     %
%\topmargin -7mm

\parskip1ex

\pagestyle{plain}

\input epsf

\date{\today}
\newcommand{\mmod}{\hspace{1mm}{\rm mod}\hspace{1mm}}
\newcommand{\lf}{\left\lfloor}
\newcommand{\rf}{\right\rfloor}
\newcommand{\norm}{|\!|}
\newcommand{\Q}{\mathbb{Q}}
\newcommand{\N}{\mathbb{N}}
\newcommand{\C}{\mathbb{C}}
\newcommand{\Z}{\mathbb{Z}}
\newcommand{\R}{\mathbb{R}}
\newcommand{\F}{\mathbb{F}}
\newcommand{\mb}{\mathbf}
\newcommand{\bigO}{{\cal O}}
\newcommand{\res}{\textrm{res}}
\newcommand{\poly}{\textrm{poly}}
\DeclareMathOperator{\wt}{wt}

\newbox\BeweisSym
\setbox\BeweisSym=\hbox{\unitlength=0.18ex%
\begin{picture}(10,10)
\put(0,0){\framebox(9,9){}}
\put(0,3){\framebox(6,6){}}
\end{picture}}
%
\newenvironment{Proof}{\noindent{\bf Proof:}$\mbox{}\;$}%
{\hfill\copy\BeweisSym\linebreak\par\noindent}
\newtheorem{Claim}{Claim}
\renewcommand{\indexname}{Was Immer Da stehen soll}

\newcommand{\marc}[1]{\textcolor{red}{\texttt{Marc: }#1}}
\newcommand{\alex}[1]{\textcolor{red}{\texttt{Alex: }#1}}
\newcommand{\map}{\textsf{map}}
\newcommand{\reduce}{\textsf{reduce}}


\begin{document}

\noindent
{\large
Project Proposals\\
within SPP 1736 -- Algorithms for Big Data\\[2ex]
\textbf{Memory-Efficient Use of Big Data in Cryptography (MemoC)}\\
}

\noindent
Marc Fischlin, TU Darmstadt\\
Alexander May, Ruhr Universit\"at Bochum\\

\noindent
{\hrulefill}

\iffalse
\subsection{Applicants}
\fi

\iffalse
\marc{Ich glaube, das alles da unten bis Summary kommt  jetzt extra}

\subsection{Topic}
Operations on encrypted data (streams), secure outsourcing of computations into the cloud, private information retrieval, secure computation of statistics on encrypted/authenticated/signed data, privacy techniques.

\subsection*{1.2 \hskip 0.3cm Thema}
Operationen auf verschl�sselten Daten(-str�men), sicheres Auslagern von Berechnungen in die Cloud, Private Information Retrieval, sichere Statistikberechnungen auf verschl�sselten/authentisierten/signierten Daten, Techniken zur Wahrung der Privatsph�re.

\subsection{Keywords}
Big Encrypted Data.

\subsection{Research area and field of work}
Theoretical computer science, complexity theory, cryptography, multi-party computation.

\subsection{Anticipated total duration}

The anticipated duration of the project is 3 years.

\subsection{Application period}

We apply for a period of 36 months funding, starting in June 2014.
\fi

\section*{Summary}
%\subsection{Summary}

\alex{Muss eventuell gek�rzt werden.}

The security of cryptographic primitives is related via reductions to the security of hard problems. A cryptographic security proof states that any successful adversary that breaks a cryptographic protocol $\Pi$ in time $t$ can be translated into an algorithm that breaks a hard problem in time $t'$. In order to instantiate cryptographic protocols $\Pi$ efficiently, one has to first find a tight security proof that closely links $\Pi$ to a hard problem, i.e. a reduction for which $t'$ is not much larger than $t$. Second, one has to make sure that any algorithm for the hard problem runs in time larger than $t'$. 
%This would then in turn show that there cannot be any adversary for $\pi$, since it would imply a faster algorithm for the hard problem.

In the past, cryptography relied on well studied hard problems from number-theory, such as factoring integers and computing discrete logarithms in elliptic curves. However, these problems stay no longer hard in the presence of quantum computers. Therefore, the NIST (National Institute of Standards and Technology) announced in autumn 2016 a call for Post-Quantum cryptographic primitives.
%, i.e. encryption, signature and key exchange. Candidates have to be submitted by Nov 2017 and then undergo a period of 3-5 of cryptanalytic research.

Most likely, the majority of reasonable candidates will be based on hard problems from coding theory and lattices, such as Learning Parities with Noise (LPN) and Learning with Errors (LWE). NIST requests well-defined security levels of 128, 192 and 256 bits classically, and 64, 80, 128 bits quantumly. This means that for instantiating LPN with e.g. 128 bit classical security, one has to make sure that any algorithm for LPN will need at least $2^{128}$ steps on a classical computer.

For making such a security claim in a reliable manner, one nowadays studies medium security levels of $50-60$ bits in practical experiments, and then extrapolates via asymptotic formulas to the desired security level. However, current algorithms for LPN and LWE for security level $b$ also require to store $2^b$ samples. This huge amount of data prevents to run experiments even for medium security levels, and thus prevents a reliable prediction of cryptographically secure key sizes.

The current project studies low-memory algorithms for problems in coding and lattice theory. We therefore concentrate on extracting a maximum amount of information of cryptographic oracles  in a streaming-based manner.



%Thus, we require data structures for encrypted data. In addition, we have to make sure that there do not exist any built-in backdoors for cryptography that allow an illegitimate person, e.g. the NSA, to decrypt the information.

%Moreover, in several scenarios it is benefical to protect only certain parts of some encrypted data. E.g., for encrypted medical data, it is useful to compute certain statistics or correlations between data elements in order to study interactions of medicaments. But this should be possible without revealing the identities of the involved patients. 

%A big open and somewhat ignored question in cryptography is whether all these operations on encrypted data can be done in a streaming model, where the encrypted data is provided as a stream rather than a randomly accessible piece of data. 



\newpage
\section*{Zusammenfassung}

\alex{to be done}




%%%%%%%%%%%%%%%%%%%%%%%%%%%%%%%%%%%%%%%%%%%%%%%%%%%%%%%%%%%%%%%%%%%%%%%%%%%%%%%%%%%%%%

\iffalse
 McKinsey Global Institute
Big data: The next frontier for innovation, competition, and productivity
May 2011 
\fi

\newpage
\section{State of the art and preliminary work}

In analyzing and instantiating cryptographic systems, one usually solely focus on time. For instance, for cryptographic reductions one does not care about the storage for oracle queries. The same is true for the study of the underlying hard problems, where often for the fastest algorithm with running time $t$, we also have a memory consumption of $t$.

While it is in general a good idea to define cryptographic security in a conservative manner, this does not properly reflect practice. While caring out $2^{60}$ operations is feasible on a medium-sized computing cluster in a reasonable amount of time, any algorithm with RAM consumption $2^{60}$ bits will not be implementable in the near future. An Internet investigation shows that nowadays the largest supercomputers\footnote{e.g. the IBM 20-Petaflops cluster installed in Sequoia, Lawrence Livermore National Laboratory, California \cite{computer}} have a RAM of at most 1.6 PB $<2^{54}$ bits. If an algorithm has to use external memory, then its running time explodes.

Hence, for estimating the security of cryptographic constructions one should settle an upper bound on the memory consumption. This in turn defines a need for finding efficient algorithms with small memory consumption. This is even more urgent for the most prominent candidates for Post-Quantum Cryptography. 

In Nov 2017, NIST will choose several cryptosystems for encryption, signature and key exchange that will then undergo a period of 3-5 of cryptanalytic research. Most likely, a majority of these systems will be based on problems from coding and lattice theory, such as Learning Parities with Noise (LPN) and Learning with Errors (LWE). But the currently best algorithms for both problems have a memory consumption which is as large as their running time, making them useless for implementing them even on medium size security levels.

\paragraph{BKW algorithm.} high memory consumption

\paragraph{Lattice Sieving and LWE.} high memory consumption



\subsubsection*{Preliminary Work of the PIs}

\paragraph{Memory-efficient LPN algorithm} describe recent EC submission

\paragraph{Asymptotic Complexity Analysis of LWE} describe DCC submission


\subsection{Project-related publications}

%\subsubsection*{Articles published by outlets with scientific quality assurance, book publications, and works accepted for publication but not yet published}

%laut DFG bei 2 Antragsstellern: 3 pro Foerderjahr, also nur 9 fuer uns.

\begingroup
\renewcommand{\section}[2]{}% eliminatinf reference section title / dirty hack

\begin{thebibliography}{99}

\bibitem{Codes12} Anja Becker, Antoine Joux, Alexander May, Alexander Meurer, 
	\textit{``Decoding Random Binary Linear Codes in $2^{n/20}$: How 1+1=0 Improves Information Set Decoding''} , In Advances in Cryptology (Eurocrypt 2012), Lecture Notes in Computer Science, Springer-Verlag, 2012.
%
\bibitem{Codes11} Alexander May, Alexander Meurer, Enrico Thomae
	\textit{``Decoding Random Linear Codes in $O(2^{0.054n})$''} , In Advances in Cryptology (Asiacrypt 2011), Lecture Notes in Computer Science, Springer-Verlag, 2011.

	

\end{thebibliography}
\endgroup

%brauchen wir nicht
%\subsubsection*{Other publications}


%\subsubsection{Patents}
%\paragraph{Pending}
%
%\paragraph{Issued}

%%%%%%%%%%%%%%%%%%%%%%%%%%%%%%%%%%%%%%%%%%%%%%%%%%%%%%%%%%%%%%%%%%%%%%%%%%%%%%%%%%%%%%


\section{Objectives and work programme}

\subsection{Anticipated total duration of the project}

The total duration time of the project is 36 months (3 years). The project duration spans from the actual project begin
(presumably June 2014) to 36 months later. Funding of the DFG is requested for the entire duration of 36 months.


\subsection{Objectives}

%The overall goal of the project is to advance the field of cryptographically-protected processing of big data by
%proposing new cryptographic solutions which match the requirements of this scenario, and by supporting an easy 
%integration of the solutions into the MapReduce framework. To this end the following sub goals should be addressed:
\begin{description}
\item[Memory Efficient LWE Classically and Quantumly.]
\item[Memory Efficient Lattice Reduction.]
\item[Practical Cryptanalysis of NIST's Post-Quantum Candidates.]
\alex{Vorschl�ge f�r Dich.}
\item[Memory Efficient Reductions?] Das h�tte Schnitt mit Eike. W�re vielleicht gar nicht so schlecht.
\item[Construction of Post-Quantum Candidate?]
\end{description}



\subsection{Work programme incl. proposed research methods}

The work program is split into parts addressed by the individual groups and by joint parts.
Project part A describes the work programme associated to Alexander May,
Project part B describes the work programme of Marc Fischlin. Part C is a joint part.
The duration of the individual work packages and their dependencies are described in
Figure~\ref{fig:WPs}.


\subsubsection{Work package A.1: Memory Efficient LWE Classically and Quantumly}
Extension of our EC submission to LWE.

\subsubsection{Work Package A.2: Memory Efficient Lattice Reduction}
Based on our ACNS paper. Integration of BKZ tricks. Finding memory-efficient sieving.

\subsection{Work Package A.3: Practical Cryptanalysis of NIST's Post-Quantum Candidates}
First hits: Frodo and new hope.

%+++ make sure it appears on page 2
\begin{figure}[t]
\begin{center}
%\ifpdf
% \includegraphics[width=0.7\textwidth]{overview.pdf}
%\else
% \includegraphics[width=0.75\textwidth]{WPs}
%\fi
\end{center}
\caption{{\small Work package durations and dependencies.}}\label{fig:WPs}
\end{figure}

\subsubsection{Work Package B.1:}

\subsubsection{Work Package B.2:}

\subsubsection{Work Package C.1:}

\subsubsection{Work Package C.2:}

\subsection{Data handling}

The results of the theoretical work is planned to be published at conferences, workshops, and in journals. This ensures
dissemination and availability of the project's results. In addition, we will use the usual electronic archives
like IACR's \texttt{eprint} service, and the PI's home pages to make full versions available in a lasting way.
As for experimental data, like software, we will also make these public, as part of the corresponding publication,
and through electronic archives like e.g. GitHub.


\subsection{Other information}

 %Please use this section for any additional information you feel is relevant which has not been provided elsewhere.
Not applicable.

\subsection{Descriptions of proposed investigations involving experiments on humans, human materials or animals}

Not applicable.

\subsection{Information on scientific and financial involvement of international cooperation partners}

Not applicable.


%%%%%%%%%%%%%%%%%%%%%%%%%%%%%%%%%%%%%%%%%%%%%%%%%%%%%%%%%%%%%%%%%%%

\section{Bibliography}


\begingroup
\renewcommand{\section}[2]{}% eliminatinf reference section title / dirty hack

\begin{thebibliography}{ABCD}

%\item[] \hspace{-0.6cm} Own work is marked with *.

%\reversemarginpar
%\marginparwidth 1pt

%\def\shortbib{0}
%\bibliographystyle{abbrv}
%\bibliography{bib}

\end{thebibliography}

\endgroup


%%%%%%%%%%%%%%%%%%%%%%%%%%%%%%%%%%%%%%%%%%%%%%%%%%%%%%%%%%%%%%%%%%%


\section{Requested modules/funds}
%Explain each item for each applicant (stating last name, first name).

\subsection{Basic Module}

\subsubsection{Funding for Staff}

We apply for the following funding for staff, following the DFG's personal rates for 2016 (DFG-Vordruck 60.12). 
Due to competitive nature of positions in Computer Science and in the area of IT security, we request
funding for full doctoral student positions (100\%). We apply for funding of two doctoral
students for the entire duration of the project. One student will be associated to TU Darmstadt (TUD)
and work on packages B.1, B.2, and C.1, the other student will be associated to Ruhr-University Bochum (RUB)
and work on packages A.1, A.2, and C.1. Student Assistents (one for each project partner) have been calculated with 10h/months
for the full duration of the project, with average costs of 11 EUR/h. They should support the implementations
in the corresponding work packages.

\quad % make more space

\noindent {\footnotesize
\begin{tabular}{|l||l|l|l||r|r|r|}
 \hline No &  Type & assoc.~to & Description & Y1 & Y2 & Y3\\
 \hline \hline
 1 & Doctoral Student, 100\%  & TUD & WP B.1, B.2,C.1 & 61,800\euro & 61,800\euro & 61,800\euro \\
 2 & Doctoral Student, 100\%  & RUB& WP A.1,A.2,C.1 & 61,800\euro & 61,800\euro & 61,800\euro\\
 3 & Student Assistent, 100\% & TUD & Implementations & 1,320\euro & 1,320\euro & 1,320\euro\\
 4 & Student Assistent, 100\% & RUB & Implementations & 1,320\euro & 1,320\euro & 1,320\euro \\
 \hline
  & Total Amount  & & & 126,240\euro & 126,240\euro & 126,240\euro\\ \hline 
\end{tabular}
}

\quad

%The doctoral students should ...

%{\em two TVL 13-position} and 
%{\em two studentische Hilfskraft position} (10 h per week).
%We apply for a 3-year funding period of all positions, starting in June 2014.\\


\subsubsection{Direct Project Costs}


\paragraph{Equipment up to EUR 10,000, Software and Consumables.}
We request an overall funding of 3,000 EUR for each applicant over the entire project duration, 6,000 EUR in total,
to carry out experiments in Amazon's commercial Elastic MapReduce. Using Amazon's platform instead of local
computing centers of both universities allows for an easier benchmarking and a smoother collaboration.

\paragraph{Travel Expenses.}
We apply for 8,000 Euro travel funding per year, i.e. for a total travel funding of 24,000 Euro.
This funding will cover expenses of the project partners for traveling between Darmstadt and Bochum
(500 EUR per year per partner), as well as attendances of potential  meetings of the SPP (500 EUR per year per 
partner). In addition, we calculate with 3000 EUR per year and partner for presenting the project results
at internationally renowned conferences and workshops (where we assume 1,500 EUR for an intercontinental conference, 
1,000 EUR for a European conference, and 500 EUR for a workshop attendance).


\iffalse
\paragraph{Visiting Researchers (excluding Mercator Fellows)}

\paragraph{Expenses for Laboratory Animals}


\paragraph{Other Costs}

\paragraph{Project-related publication expenses}


\subsection{Instrumentation}

\paragraph{Equipment exceeding Euro 10,000}


\paragraph{Major Instrumentation exceeding Euro 50,000}


\subsection{Module Temporary Position for Funding}


\subsection{Module Replacement Funding}
\fi

\iffalse %nicht relevant
4.4	Module Temporary Clinician Substitute


[Text]
4.5	Module Mercator Fellows

[Text]

4.6	Module Workshop Funding

[Text]

4.7	Module Public Relations Funding

[Text]

\fi

%%%%%%%%%%%%%%%%%%%%%%%%%%%%%%%%%%%%%%%%%%%%

\section{Project requirements}

\subsection{Employment status information}

\noindent
{\bf Marc Fischlin}, Dr.rer. nat \\
Heisenberg-W3-Professor (till June 2014, up for evaluation for tenure)\\
Postdoc TU Darmstadt (tenured, on sabbatical for Heisenberg Professorship)\\
Year of birth: 1973, Nationality: German\\[0.3cm]
Cryptography and Complexity Theory\\
Department of Computer Science\\
Technische Universit\"at Darmstadt\\
Karolinenplatz 5, 64289 Darmstadt, Germany\\[0.3ex]
Phone office: +49-(0)6151/16-3337\\
Fax office: +49-(0)6151-16-72156\\
E-Mail: marc.fischlin@cryptoplexity.de \\[0.3cm]
Privat address: Charlotte-Posenenske-Str.58, 65197 Wiesbaden\\
Phone private: +49-(0)611/9882730\\ \\



\noindent {\bf Alexander May}, Dr. rer. nat\\
W3-Professor (tenured)\\
Year of birth: 1974, Nationality German \\[0.3cm]
Faculty of Mathematics, NA 5/73\\
Ruhr-University Bochum\\
Universit\"atsstr. 150, 44801 Bochum\\[0.3cm]
Phone office: 0234/32-23261\\
Fax office: 0234/32-14430 \\
E-Mail: alex.may@rub.de \\[0.3cm]
Privat address: Im Haarmannsbusch 34, 44797 Bochum\\
Phone private: 015159-207944\\
%

% For each applicant, state the last name, first name, and employment status (including duration of contract and funding body, if on a fixed-term contract).

%[Text]  

\subsection{First-time proposal data}
%Only if applicable: Last name, first name of first-time applicant.
\vspace*{-1.5ex}
Not applicable.
%[Text]

\subsection{Composition of the project group}

%List only those individuals who will work on the project but will not be paid out of the project funds. State each person’s name, academic title, employment status, and type of funding.
\textbf{TU Darmstadt:}
\begin{itemize}
\item Prof. Dr. Marc Fischlin, Chair for Cryptography and Complexity Theory (funded through Heisenberg-Program FI 940/3-1 of DFG) 
\item Dr. Pooya Farshim (funded through Heisenberg grant FI 940-1/1)
\item Dipl.-Inform. Paul Baecher (funded through Heisenberg grant FI 940-1/1)
\item M.Sc.(Math) Tommaso Gagliardoni (funded through BMBF Security Competence Center EC SPRIDE) 
\item M.Sc.(CS), M.Sc.(IT Security) Felix G\"unther (funded through BMBF Security Competence Center EC SPRIDE) 
\item M.Sc.(Math) Giorgia Azzurra Marson  (funded through State Hesse, LOEWE center CASED) 
\item M.Sc.(CS) Arno Mittelbach (funded through State Hesse, LOEWE center CASED) 
\end{itemize}
Support: 1 secretary.

\quad

\noindent
{\bf Ruhr-University Bochum:}
\begin{itemize}
\item Prof. Dr. Alexander May, chair of the group Crypto \& IT security
\item Emmy Noether group leader Dr. Christopher Wolf (associated)
\item Dipl.-Math. Gottfried Herold 
\item Dipl.-Ing. Ilya Ozerov (funded by SPP 1307 -- Algorithm Engineering) 
\item Dipl.-Math. Elena Kirshanova (funded by GRK 1817 -- Ubiquitous Cryptography)
\item Dipl.-Ing. Stefan Hoffman (third party funding) 
\end{itemize}
Support: 1 secretary and 1 technical assistant.



\subsection{Cooperation with other researchers}

\subsubsection*{Researchers with whom you have agreed to cooperate on this project}

%We plan to continue our cooperations with the following colleagues:

\noindent
\textbf{Marc Fischlin:}
%
\begin{itemize}
\item Dr. Sebastian Gajek (NEC Research, Heidelberg)
\item Dr. \"Ozg\"ur Dagdelen/Prof. Dr. Johannes Buchmann (TU Darmstadt)
\end{itemize}

\noindent
\textbf{Alexander May:}
%
\begin{itemize}
\item Prof. Dr. Antoine Joux (University Versailles), G\"odel prize 2013, visis our group in Nov 2013 for a research stay
\item Prof. Dr. Johannes Bl\"omer (Paderborn University)
\item Prof. Dr. Christian Sohler (TU Dortmund), joint seminar ```Perlen der Theoretischen Informatik''' since 2009
\item Prof. Dr. Hans Simon (Ruhr-University Bochum)
\end{itemize}
%


\subsubsection*{Researchers with whom you have collaborated scientifically within the past three years}

\textbf{Marc Fischlin:}
\begin{itemize}
\item Professors Johannes Buchmann and Stefan Katzenbeisser (TU Darmstadt, Germany)
\item Professor Mark Manulis (University of Surrey, UK)
\item Dr.~Werner Stefan (DFKI Saarb\"ucken, Germany)
\item Jun.-Prof.~Dominique Schr\"oder (Saarland University, Germany)
\item Professors Nigel Smart, Martijn Stam, Bogdan Warinschi (University of Bristol, UK)
\end{itemize}

\noindent
\textbf{Alexander May:}
\begin{itemize}
\item Professors Eike Kiltz and Hans Simon (Ruhr-University Bochum)
\item Professor Antoine Joux (University Versailles)
\item Professor Johannes Bl\"omer (Paderborn University)
\item Professor Nigel Smart (University of Bristol, UK)
\end{itemize}


\subsection{Scientific equipment}

Besides standard equipment the project partners need to be able to carry out
the experiments with MapReduce. Since the computing center of TU Darmstadt currently
does not offer such a service, and for easier collaborations, we have chosen
Amazon's elastic MapReduce as a common platform.


\subsection{Project-relevant interests in commercial enterprises}

Not applicable.

%%%%%%%%%%%%%%%%%%%%%%%%%%%%%%%%%%%%%%%%%%
\section{Additional information}

%If applicable, please list proposals requesting major instrumentation and/or those previously submitted to a third party here.


We have not requested funding for this project from any other sources. In the event that we
submit such a request, we will inform the Deutsche Forschungsgemeinschaft immediately.

The DFG liaison officer's of TU Darmstadt and Ruhr-University Bochum will be informed of this research funding request.



%\newpage

\iffalse
%\newpage
\section{Prerequisites for carrying out the project}

For the project we do not have other funding sources. TU Darmstadt and Ruhr-University Bochum will provide all other required facilities (office, computing resources, etc.)

\subsection{Our team -- Organisational structure of our groups}
%
{\bf Ruhr-University Bochum:}
\begin{itemize}
\item Prof. Dr. Alexander May, chair of the group Crypto \& IT security
\item Emmy Noether group leader Dr. Christopher Wolf (associated)
\item Dipl.-Math. Gottfried Herold 
\item Dipl.-Ing. Ilya Ozerov (currently financed by SPP 1307 -- Algorithm Engineering) 
\item Dipl.-Math. Elena Kirshanova (currently financed by GRK 1817 -- Ubiquitous Cryptography)
\item Dipl.-Ing. Stefan Hoffman (third party funding) 
\end{itemize}

Support: 1 secretary and 1 technical assistant

\subsection{Cooperation with other scientists}

We plan to continue our cooperations with the following colleagues.
%
\begin{itemize}
\item Prof. Dr. Antoine Joux (University Versailles), G�del prize 2013, visis our group in Nov 2013 for a research stay
\item Prof. Dr. Johannes Bl�mer (Paderborn University)
\item Prof. Dr. Christian Sohler (TU Dortmund), joint seminar ```Perlen der Theoretischen Informatik''' since 2009
\item Prof. Dr. Hans Simon (Ruhr-University Bochum)
\end{itemize}
%

\subsection{Scientific equipment}

fully existing

\subsection{Running costs for material}

not necessary

\subsection{Other requirements}

none


%%%%%%%%%%%%%%%%%%%%%%%%%%%%%%%%%%%%%%%%%%%%%%%%%%%%%%%%%%%%%%%%%%%%%%

%\newpage
\section{Declarations}


We have not requested funding for this project from any other sources. In the event that we
submit such a request, we will inform the Deutsche Forschungsgemeinschaft immediately.

The DFG liaison officer's of TU Darmstadt's and Ruhr-University Bochum will be informed of this research funding request.


%%%%%%%%%%%%%%%%%%%%%%%%%%%%%%%%%%%%%%%%%%%%%%%%%%%%%%%%%%%%%%%%%%%%%%

\section{Signature}

\vspace{2cm}

TU Darmstadt, 30.09.2013 \hspace{4cm} (Marc Fischlin) \hspace{4cm} Bochum, 15.03.2011 \hspace{4cm} (Alexander May) \\


%%%%%%%%%%%%%%%%%%%%%%%%%%%%%%%%%%%%%%%%%%%%%%%%%%%%%%%%%%%%%%%%%%%%%%

\section{Attachments}

\begin{itemize}
\item CV of the applicants
\item Paper: A. May, I. Ozerov, ``On Merging Lists Consistently: Solving Subset Sum in $2^{0.287n}$'', currently in submission.
\end{itemize}

%%%%%%%%%%%%%%%%%%%%%%%%%%%%%%%%%%%%%%%%%%%%%%%%%%%%%%%%%%%%%%%%%%%%%%

\newpage
\subsection{Prof. Dr. rer. nat. Alexander May}

born on March 22, 1974 in Friedberg (Germany)\\
Ruhr-University Bochum\\
Chair for Cryptology and IT-Security\\
Faculty of Mathematics\\
Phone: +49 234 32 23261\\
Email: alex.may@rub.de\\

\subsubsection*{Education \& Professional Experience}
\begin{tabular}{ll}
{10.93 -- 07.99} & Computer Science study, J.W. Goethe-Universit\"at in Frankfurt/Main \\
{09.99 -- 02.00} & PhD, Computer Science Department, ETH Z\"urich \\
{03.00 -- 12.03} & PhD, Computer Science Department, Paderborn University \\
{01.04 -- 09.05} & Post-Doc in DFG-Priority Programme \\
%
& {``Sicherheit in der Informations- und
Kommunikationstechnik''}\\
%
{10.05 -- 09.07} & Juniorprofessor, Cryptographic Protocols, TU Darmstadt \\
{since 10.07} & W3-Professor, Cryptology and IT-Security, Ruhr-University Bochum \\
\end{tabular}




\vskip 0.3cm



%\subsubsection*{Academic Honors and Awards}

%\begin{tabular}{ll}
%
%2009 & Best Paper Award PKC together with Maike Ritzenhofen for the paper \\
%& \textit{``Implicit Factoring: On Polynomial Time Factoring Given Only an Implicit Hint''} \\
%2006 & Best Paper Award PKC together with Daniel Bleichenbacher for the paper \\
%&  {\it ``New Attacks on RSA with Small Secret CRT-Exponents''} \\
%2005 & Price of Faculty Electrical Engineering, Computer Science and Mathematics\\
%& for the PhD thesis  {\it ``New RSA Vulnerabilities Using Lattice Reduction Methods''} \\
%& \\
%2007 & Price of Student Body Computer Science at TU Darmstadt for the best lecture\\
%2006 & Price of Student Body Computer Science at TU Darmstadt for the best lecture\\
%2004 & Weierstra\ss-Price of Faculty Electrical Engineering, Computer Science and Math.\\
%&  at Paderborn University for excellent teaching\\
%\end{tabular}


\subsubsection*{Research Grants Relevant for the Application}

\begin{itemize}
\item DFG-GRK 1817 Ubiquitous Cryptography, ``Outsourced Computation'', since 2013
\item ERC Starting Grant 307952 (Partner), ``Fast and Sound Cryptography'', since 2012 
\item DFG-SPP 1307 Algorithm Engineering, ``Algorithms for Subset Sum, Lattices and Linear Codes'', 2011-13
\item DFG RUB-RS (ExIni), ``Coding-Based Cryptanalysis'', 2009-12
\item DFG, ``Lattice-Based Solving of Polynomial Equations'', 2007-10
\item EU, ECRYPT II -- European Network of Excellence in Cryptology, since 2008
\item DFG-SPP 1079 IT Security, ``Lattice Attacks on RSA'', 2003-2005
\end{itemize}

\subsection*{Brief Description of the Chair for Cryptology and IT-Security}
The research group interests cover all aspects of modern cryptology, algorithmic number theory and discrete mathematics.
The chair's special focus lies on algorithmic aspects of cryptography.
Our spectrum of cryptographic applications covers problems from factoring, discrete logarithms, lattice and coding theory, diophantine equations and quantum algorithms.


\vskip 0.3cm


\subsubsection*{Five Publications most Relevant for the Application}
%\footnote{Die wichtigsten Beitr\"age sind \textbf{fett} gedruckt.}}

\vspace{0.0cm}


%\textbf{Book chapter  (reviewed)}

%\begin{itemize}
%\item \textit{``Using LLL-Reduction for Solving RSA and Factorization Problems: A Survey''} \\
%In ``The LLL Algorithm --Survey and Applications'', Editors Phong Nguyen and Brigitte Vall\'ee, Springer, 2010.
%\end{itemize}



%\noindent \textbf{Journals (reviewed)}

%\begin{itemize}
%\item \textit{``Deterministic Polynomial Time Equivalence of Computing
%the RSA Secret Key and Factoring''}  with Jean-S\'ebastien Coron \\
%Journal of Cryptology, 2007.

%\item \textit{``Perspectives for Cryptographic Long-Term Security''} with Johannes Buchmann and Ulrich Vollmer \\
%Communications of the ACM, Vol. 94(9), 2006.

%\item \textit{``Krypto 2020''} with Johannes Buchmann, Erik Dahmen and Ulrich Vollmer \\
%KES -- The Information Security Journal, Vol. 5, 2005.

%\end{itemize}

%\noindent \textbf{Conference Papers (reviewed)}

%\vspace{0.8cm}

\begin{itemize}
\item Saqib Kakvi, Eike Kiltz, Alexander May, \textit{``Certifying RSA''},  In Advances in Cryptology (Asiacrypt 2012), Lecture Notes in Computer Science, Springer-Verlag, 2012.
\item Anja Becker, Antoine Joux, Alexander May, Alexander Meurer, 
	\textit{``Decoding Random Binary Linear Codes in $2^{n/20}$: How 1+1=0 Improves Information Set Decoding''} , In Advances in Cryptology (Eurocrypt 2012), Lecture Notes in Computer Science, Springer-Verlag, 2012.
\item Alexander May, Alexander Meurer, Enrico Thomae
	\textit{``Decoding Random Linear Codes in $O(2^{0.054n})$''} , In Advances in Cryptology (Asiacrypt 2011), Lecture Notes in Computer Science, Springer-Verlag, 2011.
\item Jake Loftus, Alexander May, Nigel P. Smart, Frederik Vercauteren
	\textit{``On CCA-Secure Somewhat Homomorphic Encryption''} , In Selected Areas in Cryptography (SAC 2011), Lecture Notes in Computer Science, Springer-Verlag, 2011.
\item Wilko Henecka, Alexander May, Alexander Meurer
	\textit{``Correcting Errors in RSA Private Keys''}, In Advances in Cryptology (Crypto 2010), Lecture Notes in Computer Science, Springer-Verlag, 2010.
\end{itemize}

\fi

\end{document}


\iffalse DFG Vorlage

Project Description - Project Proposals


[First name  last name, city of all applicants] 
 





2	Objectives and work programme
2.1	Anticipated total duration of the project

[Text]
2.2	Objectives

[Text]
2.3	Work programme incl. proposed research methods
For each applicant

[Text]
2.4	Data handling

[Text]

2.5	Other information
Please use this section for any additional information you feel is relevant which has not been provided elsewhere.

[Text]

2.6	Descriptions of proposed investigations involving experiments on humans, human materials or animals

[Text]

2.7	Information on scientific and financial involvement of international cooperation partners

[Text]



3	Bibliography

[Text]




\fi


\iffalse
%===============
I Priority Programme
For individual project proposals within an established Priority Programme, note that the fund-ing duration (part A of the proposal) and the funding periods are specified in the call for pro-posals.
For the project description (part B of the proposal) note the following:
Each proposal must be accompanied by a description of how the project is integral to the Priority Programme, both in terms of subject matter and organisation. This includes a de-scription of the cooperation with others participating within the Priority Programme. The en-visaged realisation of the project in cooperation with other applicants may be demonstrated in particular by the joint training of early career researchers, or the use of methods by multi-ple projects as part of a network.
DFG form 54.01 - 04/13 page 11 of 14
Deutsche Forschungsgemeinschaft
Kennedyallee 40 ∙ 53175 Bonn ∙ postal address: 53170 Bonn
phone: + 49 228 885-1 ∙ fax: + 49 228 885-2777 ∙ postmaster@dfg.de ∙ www.dfg.de DFG
All applicants involved in submitting a proposal within an established Priority Programme are obliged to promptly provide the overall coordinator with all of the information necessary for drawing up the interim reports and the final report for the Priority Programme.
\fi