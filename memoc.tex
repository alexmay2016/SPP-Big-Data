\documentclass[12pt]{article}
\usepackage[a4paper,hmargin=1in,vmargin=1.25in]{geometry}

%\usepackage{german}
\usepackage{latexsym,wrapfig}
\usepackage{amsmath,amssymb}
\usepackage{epsfig,color,colordvi}
\usepackage{graphics}
\usepackage{theorem}
\usepackage{pstricks}
\usepackage{amsfonts}
\usepackage[T1]{fontenc}
\usepackage{eurosym}

%\textwidth 16.1cm      %changea4.tex
%\textheight 170mm      %
%\textheight 223mm      %
%\evensidemargin 0mm    %
%\oddsidemargin 0mm     %
%\topmargin -7mm

\parskip1ex

\pagestyle{plain}

\input epsf

\date{\today}
\newcommand{\mmod}{\hspace{1mm}{\rm mod}\hspace{1mm}}
\newcommand{\lf}{\left\lfloor}
\newcommand{\rf}{\right\rfloor}
\newcommand{\norm}{|\!|}
\newcommand{\Q}{\mathbb{Q}}
\newcommand{\N}{\mathbb{N}}
\newcommand{\C}{\mathbb{C}}
\newcommand{\Z}{\mathbb{Z}}
\newcommand{\R}{\mathbb{R}}
\newcommand{\F}{\mathbb{F}}
\newcommand{\mb}{\mathbf}
\newcommand{\bigO}{{\cal O}}
\newcommand{\res}{\textrm{res}}
\newcommand{\poly}{\textrm{poly}}
\DeclareMathOperator{\wt}{wt}

\newbox\BeweisSym
\setbox\BeweisSym=\hbox{\unitlength=0.18ex%
\begin{picture}(10,10)
\put(0,0){\framebox(9,9){}}
\put(0,3){\framebox(6,6){}}
\end{picture}}
%
\newenvironment{Proof}{\noindent{\bf Proof:}$\mbox{}\;$}%
{\hfill\copy\BeweisSym\linebreak\par\noindent}
\newtheorem{Claim}{Claim}
\renewcommand{\indexname}{Was Immer Da stehen soll}

\newcommand{\marc}[1]{\textcolor{red}{\texttt{Marc: }#1}}
\newcommand{\alex}[1]{\textcolor{red}{\texttt{Alex: }#1}}
\newcommand{\map}{\textsf{map}}
\newcommand{\reduce}{\textsf{reduce}}

\newcommand{\url}[1]{\texttt{#1}}

\begin{document}

\noindent
{\large
Project Proposals\\
within SPP 1736 -- Algorithms for Big Data\\[2ex]
\textbf{Memory-Efficient Use of Big Data in Cryptography (MemoC)}\\
}

\noindent
Marc Fischlin, TU Darmstadt\\
Alexander May, Ruhr Universit\"at Bochum\\

\noindent
{\hrulefill}

\iffalse
\subsection{Applicants}
\fi

\iffalse
\marc{Ich glaube, das alles da unten bis Summary kommt  jetzt extra}

\subsection{Topic}
Operations on encrypted data (streams), secure outsourcing of computations into the cloud, private information retrieval, secure computation of statistics on encrypted/authenticated/signed data, privacy techniques.

\subsection*{1.2 \hskip 0.3cm Thema}
Operationen auf verschl�sselten Daten(-str�men), sicheres Auslagern von Berechnungen in die Cloud, Private Information Retrieval, sichere Statistikberechnungen auf verschl�sselten/authentisierten/signierten Daten, Techniken zur Wahrung der Privatsph�re.

\subsection{Keywords}
Big Encrypted Data.

\subsection{Research area and field of work}
Theoretical computer science, complexity theory, cryptography, multi-party computation.

\subsection{Anticipated total duration}

The anticipated duration of the project is 3 years.

\subsection{Application period}

We apply for a period of 36 months funding, starting in June 2014.
\fi

\section*{Summary}
%\subsection{Summary}

\alex{Muss eventuell gek�rzt werden.}

The security of cryptographic primitives is related via reductions to the security of hard problems. A cryptographic security proof states that any successful adversary that breaks a cryptographic protocol $\Pi$ in time $t$ can be translated into an algorithm that breaks a hard problem in time $t'$. In order to instantiate cryptographic protocols $\Pi$ efficiently, one has to first find a tight security proof that closely links $\Pi$ to a hard problem, i.e. a reduction for which $t'$ is not much larger than $t$. Second, one has to make sure that any algorithm for the hard problem runs in time larger than $t'$. 
%This would then in turn show that there cannot be any adversary for $\pi$, since it would imply a faster algorithm for the hard problem.

In the past, cryptography relied on well studied hard problems from number-theory, such as factoring integers and computing discrete logarithms in elliptic curves. However, these problems stay no longer hard in the presence of quantum computers. Therefore, the NIST (National Institute of Standards and Technology) announced in autumn 2016 a call for Post-Quantum cryptographic primitives.
%, i.e. encryption, signature and key exchange. Candidates have to be submitted by Nov 2017 and then undergo a period of 3-5 of cryptanalytic research.

Most likely, the majority of reasonable candidates will be based on hard problems from coding theory and lattices, such as Learning Parities with Noise (LPN) and Learning with Errors (LWE). NIST requests well-defined security levels of 128, 192 and 256 bits classically, and 64, 80, 128 bits quantumly. This means that for instantiating LPN with e.g. 128 bit classical security, one has to make sure that any algorithm for LPN will need at least $2^{128}$ steps on a classical computer.

For making such a security claim in a reliable manner, one nowadays studies medium security levels of $50-60$ bits in practical experiments, and then extrapolates via asymptotic formulas to the desired security level. However, current algorithms for LPN and LWE for security level $b$ also require to store $2^b$ samples. This huge amount of data prevents to run experiments even for medium security levels, and thus prevents a reliable prediction of cryptographically secure key sizes.

The current project studies low-memory algorithms for problems in coding and lattice theory. We therefore concentrate on extracting a maximum amount of information of cryptographic oracles  in a streaming-based manner.



%Thus, we require data structures for encrypted data. In addition, we have to make sure that there do not exist any built-in backdoors for cryptography that allow an illegitimate person, e.g. the NSA, to decrypt the information.

%Moreover, in several scenarios it is benefical to protect only certain parts of some encrypted data. E.g., for encrypted medical data, it is useful to compute certain statistics or correlations between data elements in order to study interactions of medicaments. But this should be possible without revealing the identities of the involved patients. 

%A big open and somewhat ignored question in cryptography is whether all these operations on encrypted data can be done in a streaming model, where the encrypted data is provided as a stream rather than a randomly accessible piece of data. 



\newpage
\section*{Zusammenfassung}

\alex{to be done}




%%%%%%%%%%%%%%%%%%%%%%%%%%%%%%%%%%%%%%%%%%%%%%%%%%%%%%%%%%%%%%%%%%%%%%%%%%%%%%%%%%%%%%

\iffalse
 McKinsey Global Institute
Big data: The next frontier for innovation, competition, and productivity
May 2011 
\fi

\newpage
\section{State of the art and preliminary work}

Big data processing in cryptography often refers to the efficiency of mounting attacks. Typically, if an adversary gets access to a sufficient amount of data and can provide enough resources, then the sought-after information becomes available. Remarkably, in most cases cryptographers focus on the time requirements of attacks in order to assess the security of solutions. Yet, the memory consumption for both storing large amounts of data and for executing attacks is an important factor, too.%
\footnote{A concrete example where this fact has been brought back to the center of attention is the initiative of the US National Security Agency (NSA) to build the so-called Utah Data Center for mass storage of data \cite{Utah}.}


\paragraph{The Impact of Memory Consumption.}
For analyzing and instantiating cryptographic systems one usually  focuses on the aspect of adversarial running time. For instance, for cryptographic reductions one usually does not care about memory requirements, such as for the storage for oracle queries and answers. The same is true for the study of the underlying hard problems. The reason is that, often, the fastest algorithm with running time $t$ also has a memory consumption of roughly $t$. 

In cryptography, however, the space requirement for an attack can be a significant factor.
%While it is in general a good idea to define cryptographic security in a conservative manner, this does not properly reflect practice. 
While performing $2^{60}$ operations today is considered to be feasible, even on a medium-sized computing cluster in a reasonable amount of time, any algorithm with RAM consumption $2^{60}$ bits will not be implementable in the near future. An Internet investigation shows that nowadays the largest supercomputers\footnote{e.g. the IBM 20-Petaflops cluster installed in Sequoia, Lawrence Livermore National Laboratory, California \cite{computer}} have a RAM of at most 1.6 PB $<2^{54}$ bits. If an algorithm has to use external memory, then its running time slows down significantly.

Hence, for estimating the security of cryptographic constructions one should also consider an upper bound on the memory consumption. This in turn defines a need for finding efficient algorithms with small memory consumption. 

\paragraph{The Impact on LPN and LWE.}
In this proposal we combine the question of memory consumption with a recent development in cryptography, due to the potential advances in quantum computing. Nowadays the question pops up which cryptography can still be considered to be suitable to protect data, since classical problems like RSA or discrete logarithms will become insecure once quantum computers reach maturity.

%This is even more urgent for the most prominent candidates for Post-Quantum Cryptography. 

In November 2017, the National Institute of Standards and Technology (NIST) will open a call for candidates of cryptosystems for encryption, signature and key exchange, which are presumably immune to quantum attacks. The candidates will then undergo a period of 3-5 years of cryptanalytic research, before a recommendation is made. Most likely, a majority of these systems will be based on problems from coding and lattice theory, such as Learning Parities with Noise (LPN) and Learning with Errors (LWE). But the currently best algorithms for both problems have a memory consumption which is as large as their running time, making them useless for implementing them even on medium size security levels.


\paragraph{BKW algorithm.}

\paragraph{Lattice Sieving.}



\paragraph{Deployment of LPN and LWE in Key Exchange Protocols.}
Analogously to the problem of determining the necessary resources for mounting attacks, in order to make recommendations for secure choices for the underlying problems, we are interested in the security when these problems are deployed in more complex protocols.
%influence of the deployment of the problems like LWE and LPN. 
We are especially interested in securing communication data and the question how we can prevent attackers to decipher data which is stored now and potentially sifted through later, e.g., once quantum computers are available. The connection to the first part of the proposal is via the estimated hardness of potentially quantum-resistant problems such as LPN and LWE. 
%The research in that part will suggest parameter choices withstanding Big Data engineering efforts, such that we can build secure systems and recommend appropriate parameter choices. 

Securing communication data between two parties typically consists of composing a so-called key exchange protocol with a secure channel protocol.  With the secure key exchange protocol the participants establish a shared cryptographic key, which should be known only to them. Then, this key is used in a secure channel protocol to send the actual data in a confidential and authentic way. 

The focus for securing communication against advanced attacks, especially against quantum cryptanalysis, currently lies on the key exchange part. The reason is twofold. The first reason is that most of the practically deployed key exchange protocols (such as the ones used in TLS~1.2 \cite{TLS12} and the future TLS~1.3 \cite{TLS13}) rely on number-theoretic problems which are highly amenable to quantum attacks. The best known examples are the paramount Diffie-Hellman based protocols which can be broken in polynomial time by quantum attackers. 


Concerning efforts to build quantum-resistant key exchange protocol, this area has gained quite some momentum, culminating in proposals to derive such protocols from Ring-LWE \cite{P14,BCNS15,ADPS16} and LWE \cite{BCDMNNRS16}. For the scheme in \cite{ADPS16}, the ``New Hope'' key exchange protocol, Google recently announced to experiment with the scheme in its Chrome browser \cite{Google}. Another recent proposal for a potentially quantum-resistant key echange protocol is based on ideal lattices \cite{ZZDSD15}.

\paragraph{Shortcomings of current LPN-/LWE-based Key Exchange Protocols.}

Unfortunately, most of the proposals have at least one shortcoming. For example, the analysis of New Hope \cite{ADPS16} relies on the classical random oracle model, although it has been argued in \cite{BDFLSZ11} that quantum access to this idealized primitive should be preferred in such settings. Others, such as \cite{BCNS15,BCDMNNRS16} provide security in the ACCE notion of Jager et al.~\cite{JKSS12}, arguing security when the key exchange protocol is composed with an atomic channel. This at the moment excludes a modular analysis if instead a stream-based channel is used (as possible in TLS) \cite{FGMP15}. Moreover, all the analyses follow the classical choice of investigating primarily the run time and success probability of adversaries, mainly neglecting the memory consumption in the  reduction.


Most of the aforementioned works  start with the unauthenticated key exchange setting, where parties and transmissions are not authenticated. They argue that this can be accomplished later by adding signatures. Here, however, it is unclear if breaking the authentication with a quantum computer afterwards could not lead to a break of current executions (since the secrets may still be in use later). Furthermore, in some cases this requires some changes to the structure, such that for example \cite{BCNS15,BCDMNNRS16} are, strictly speaking, not compatible to TLS.


Another shortcoming of the current proposals is that it they may not be easy to integrate into the upcoming TLS~1.3 protocol. While \cite{BCNS15,ADPS16,BCDMNNRS16} argue how this can be done for TLS~1.2 and provide impressive implementation and performance details, the next TLS~1.3 version adds significantly enhanced functional properties such as zero round-trip time steps. In the Diffie-Hellman setting this often necessitates to switch to other number-theoretic problems, such as the PRF-ODH assumption \cite{JKSS12}. It is unclear what this would mean for LWE- and LPN-based protocols, and if adopting a new assumption, for the memory consumption for this related problem.




\subsubsection*{Preliminary Work of the PIs}

\paragraph{Memory-efficient LPN algorithm}

\paragraph{Asymptotic Complexity Analysis of LWE}
\paragraph{Key Exchange.}
We have extensively contributed to fundamental question about key exchange models \cite{BFWW11,FG14}, dealing with composability and multi-stage key exchange protocols. We have provided analyses of the (Diffie-Hellman based) TLS protocols \cite{BFSWW13,DFGS15}, both for versions 1.2 and the (candidates for) 1.3. We have also investigated the special modes and properties of the candidates for TLS version 1.3~\cite{DFGS15,FGSW16}.

\paragraph{Cryptographic Reductions.}
Since our research area is complexity-based cryptography, reductions are our main tool in conducting security proofs of cryptographic protocols and appear in the majority of our works. Concerning reductions themselves we have extensive expertise about notions of reductions \cite{F12,BBF13}. All the aforementioned works about key exchange involve reductions in the domain of key exchange.





\subsection{Project-related publications}

%\subsubsection*{Articles published by outlets with scientific quality assurance, book publications, and works accepted for publication but not yet published}

%laut DFG bei 2 Antragsstellern: 3 pro Foerderjahr, also nur 9 fuer uns.

\begingroup
\renewcommand{\section}[2]{}% eliminatinf reference section title / dirty hack

\begin{thebibliography}{99}

\bibitem{Codes12} Anja Becker, Antoine Joux, Alexander May, Alexander Meurer, 
	\textit{``Decoding Random Binary Linear Codes in $2^{n/20}$: How 1+1=0 Improves Information Set Decoding''} , In Advances in Cryptology (Eurocrypt 2012), Lecture Notes in Computer Science, Springer-Verlag, 2012.
%
\bibitem{Codes11} Alexander May, Alexander Meurer, Enrico Thomae
	\textit{``Decoding Random Linear Codes in $O(2^{0.054n})$''} , In Advances in Cryptology (Asiacrypt 2011), Lecture Notes in Computer Science, Springer-Verlag, 2011.

	

\end{thebibliography}
\endgroup

%brauchen wir nicht
%\subsubsection*{Other publications}


%\subsubsection{Patents}
%\paragraph{Pending}
%
%\paragraph{Issued}

%%%%%%%%%%%%%%%%%%%%%%%%%%%%%%%%%%%%%%%%%%%%%%%%%%%%%%%%%%%%%%%%%%%%%%%%%%%%%%%%%%%%%%


\section{Objectives and work programme}

\subsection{Anticipated total duration of the project}

The total duration time of the project is 36 months (3 years). The project duration spans from the actual project begin (presumably June 2017) to 36 months later. Funding of the DFG is requested for the entire duration of 36 months.


\subsection{Objectives}


The overall goal of the project is to advance the field of cryptographically-protected processing of big data by
proposing new cryptographic solutions which match the requirements of this scenario, and by supporting an easy 
integration of the solutions into the MapReduce framework. To this end the following sub goals should be addressed:
\begin{description}
 \item[Fundamental Operations for Encrypted Data:]
 The first sub goal is to provide tailor-made solutions for a quite large set of operations, in order
 to preserve privacy. The starting set are relational operations such as union, projection, and intersection. Here
 the problem is to deal with duplicates (which need to be removed for union and projection), relating to the 
 deduplication problem for encrypted cloud storage \cite{dedup}. We also investigate problems related to 
 basic problems such as ``grepping'' and graph algorithms like PageRank (see \cite{LD10}), aiming to exploit special encryption
 schemes like deterministic encryption for performance.
% machine learning,
% in particular, how to implement operations based on computing Euclidean distances such as $k$-means or
% Locally Weighted Linear Regression \cite{ml}. 
Prototypes will be implemented and tested in MapReduce.
 
 \item[Fundamental Operations for Authenticated Data:]
 This part of the project shall design digital signature schemes and message authentication scheme which are 
 compatible with known MapReduce applications. As explained, general solutions for the authentication problem
 are still missing ---unlike for the case of encryption--- such that we first aim at specific solutions.
 Our goal is to provide efficient authentication solutions in order to combine them with the
 specific solutions for encryption in the aforementioned case. Since some cases like grepping should be easy,
 our goal is also to investigate some problems related to machine learning,
 in particular, how to implement operations based on computing Euclidean distances such as $k$-means or
 Locally Weighted Linear Regression \cite{ml}. 
   Again, solutions will be integrated and tested into 
 MapReduce.
 
\item[Feasibility of outsourced computation on encrypted data:] 
While the previous sub goals aim at specific problems, another goal is to provide general solutions.
Hence, based on the framework of functional encryption identify functionalities which can be efficiently realized on encrypted data in a cloud/outsourced computing scenario. Implement and test those functionalities in MapReduce. 

\item[Cryptography in the streaming model:] Transfer known cryptographic security notions to a streaming/highly parallelized model, where one does not have random access to the whole data. Moreover, the restriction is to (approximately) extract useful information in sublinear/polylogarithmic running time with very limited storage. To test our algorithms, we will choose as a framework Google's MapReduce. 

\item[Enhance cryptographic certification properties:] 
Since cryptographic means seem to be the only solution for protecting our privacy against security agencies/industrial espionage, we have to make sure that cryptography is designed in a way that does not offer any backdoors by publicly verifiable process. It is important that this certification is compatible with existing cryptographic infrastructure, i.e. that it does not require to renew all existing keys.

\end{description}

Besides the scientific advancements we expect our result of be of social and political importance in light of the recent 
revelations around the NSA affair.


\subsection{Work programme incl. proposed research methods}

The work program is split into parts addressed by the individual groups and by joint parts.
Project part A describes the work programme associated to Marc Fischlin,
Project part B describes the work programme of Alexander May. Part C is a joint part.
The duration of the individual work packages and their dependencies are described in
Figure~\ref{fig:WPs}.


\newcounter{workpackage}
\setcounter{workpackage}{1}

\paragraph{Workpackage \Alph{workpackage}.1}


\paragraph{Workpackage \Alph{workpackage}.2}

text

\stepcounter{workpackage}


\paragraph{Workpackage \Alph{workpackage}.1}

\paragraph{Workpackage \Alph{workpackage}.2}

text

\stepcounter{workpackage}


\refstepcounter{workpackage}\label{mf:wp:eins}

\paragraph{Work package \theworkpackage.1: Memory-efficient Reductions}

The first task is to explore the memory efficiency of the reductions in \cite{P14,BCNS15,ADPS16,BCDMNNRS16}. Such a reduction $\reduc$ from the key exchange protocol takes an adversary $\adv_\text{KE}$ against the protocol, and turns this into an algorithm $\adv_\text{LWE}=\reduc^{\adv_\text{KE}}$ against the underlying algorithmic problem; in the example here for LWE.
Since one usually considers key exchange protocols in the multi-instance setting of Bellare and Rogaway \cite{BR93}, the reduction usually needs to simulate the multiple instances of the key exchange protocol, at the expense of a significant storage requirement, e.g., if $2^{20}$ instances are running concurrently. This has major impact on the assumed hardness of the underlying problem.

We are therefore interested in the exact effect of the reduction's requirements on the suggested parameter choice. This requires us to determine the exact bounds (in particular, in terms of space) and link them to the findings of Work package \marc{XXXX; link auf ein vorheriges WP}. 
We also investigate if we can devise better reductions or bounds, e.g., in the sequential execution model.


\paragraph{Work package \theworkpackage.2: Analysis of Underlying LWE-Problems.}

In this work package we investigate the hardness of the underlying problem(s) proposed in Work packages {\ref{mf:wp:zwei}}.1 and \ref{mf:wp:zwei}.2. This investigation covers several aspects. First, we verify with the results of Work packages \marc{XXXX; link auf ein vorheriges WP} how hard the problem itself seems to be. Secondly, we try to relate the new problem (via memory-efficient reductions) to the standard LWE problem, or try to show that the problem is strictly stronger, by giving a black-box separation result.

%--
\refstepcounter{workpackage}\label{mf:wp:zwei}

\paragraph{Work package \theworkpackage.1: Adaptation to TLS~1.2.}

In this work packages we adopt ideas from the TLS~1.2 concept \cite{TLS12} to the suggested LWE-based key exchange protocol designs. The main step is to investigate if one can remove the late signatures in \cite{BCNS15} which make the solution slightly TLS-non-conforming. As discussed in \cite{BCNS15} such a change would most likely imply to switch to a PRF-ODH like assumption, allowing the adversary to mount an active attack against the underlying LWE problem. As pointed out by Peikert \cite{P14}, the problem seems to be easy under such active attacks, though. 

Fortunately, not all is lost: For TLS~1.2 the analysis of Jager et al.~\cite{JKSS12} requires only a very limited form of active attacks in which the adversary can make a single chosen queries only. For such active attacks the LWE problem may still be hard. As an alternative, or second step, we consider other designs possibilities for the key exchange protocol, thwarting this problem by design.

If possible, solutions should be implemented and compared to existing ones.

\paragraph{Work package \theworkpackage.2: Adaptation to TLS~1.3.}

This work packages looks into the possibility to adapt the ideas of previous LWE-based proposals to (the current draft of) TLS~1.3~\cite{TLS13}. Since TLS~1.3 will be fundamentally different from TLS~1.2 this at foremost requires us to check if the current solutions can be transferred at all. 

Next, we address the question if we can augment existing protocols by a 0RTT mode where one derives a fresh key without interaction by consulting previous communication data. As explained above, this presumably requires an even stronger PRF-ODH like assumption, where the adversary can make many active queries. Here the question which should be addressed is if one needs to make some restriction on the number of key exchange sessions in which material is re-used.

If possible, solutions should be implemented and compared to existing ones.





\subsubsection{Work package A.1: Fundamental Operations on Encrypted Data}

To achieve the goal of designing efficient solutions for the basic problems for 
union, projection, and intersection, and the text-based and graph-based problems we
investigate the potential of deterministic encryption \cite{BBN07}. Informally,
deterministic encryption ensures that each message has a unique ciphertext, supporting
equality checks straightforwardly, but such that basically nothing else about ciphertexts is leaked. 
The disadvantage is, however, that the message needs to be highly entropic in order to guarantee
security. 

The entropy requirement infringes with basic operations like grepping expressions in a large file. The reason
is that we cannot individually encrypt short chunks of the file and uphold confidentiality, as the entropy
guarantees may be to given anymore for short parts. The same holds for other operations like computing
the intersection. Hence, new ways to use deterministic encryption must be found. The approach we plan to investigate
is to combine deterministic encryption with a mild form of homomorphic encryption. We may
extract a cryptographic key $k$ from the big data (with high entropy), pseudorandomly derive keys $k_1,k_2,\dots$
for each chunk of the data, and then encrypt deterministically the chunks of the data, where we append the corresponding 
key to the data chunk. To search within a chunk one encrypts the key word and the starting key $k_1$ and then
``moves'' (and homomorphically updates!) this ciphertext across the chunks.
The challenge here is to find a deterministic encryption scheme and a pseudorandom generator which are 
compatible and allow to update. Here the scheme by Boldyreva et al.~\cite{BSN08} serves as our starting point,
but if necessary we may also investigate the potential of fully homomorphic encryption instead.

For the intersection and the similar relational task we may use similar ideas, although here different problems arise
due to the requirement to remove duplicates. This is also known as the deduplication problem in cloud storage \cite{dedup}.
Some solutions have been presented in \cite{dedup,dedup2}. We plan to investigate how to improve these solutions,
especially in light of incrementality, i.e., to compute solutions faster if the data is updated only slightly, and
to incorporate them into the MapReduce Framework. The final step in this work package is to investigate
the possibility to derive tailored solutions for graph-based problems, where PageRank serves as an entry point.


\subsubsection{Work Package A.2: Fundamental Operations for Authenticated Data}

In this work package we provide solutions for processing big data authentically. As mentioned before, 
since currently the development of omnipotent authentication methods lacks behind, compared to fully homomorphic
encryption, and the goal of this project is to make cryptographic methods applicable to MapReduce, our work
here focuses on designing specific solutions for concrete problems. The solutions should be combined with
the construction of work package A.1.

As a first step we will investigate
the applicability of known homomorphic signature schemes like \cite{P2EC:BonFre11,P2PKC:Freeman12,P2PKC:CatFioWar12}
to the basic relational problems like aggregation; problems like intersection at first glance seem to be straightforward.
For the graph-based problems like PageRank we will first explore the ability to use transitive signatures \cite{P2RSA:MicRiv02b}. 
We expect that this will not be applicable in all cases, such that we will develop new signature schemes if necessary.
For the applications from machine learning like Euclidean distances such as $k$-means or
 Locally Weighted Linear Regression we definitely need to devise new schemes. Our starting point will be schemes
 for related problems. All solutions will be matched to MapReduce, of course.

%+++ make sure it appears on page 2
\begin{figure}[t]
\begin{center}
%\ifpdf
% \includegraphics[width=0.7\textwidth]{overview.pdf}
%\else
% \includegraphics[width=0.75\textwidth]{WPs}
%\fi
\end{center}
\caption{{\small Work package durations and dependencies.}}\label{fig:WPs}
\end{figure}

\subsubsection{Work Package B.1: Feasibility of Outsourced computation on encrypted data}
The starting point of our work programme in this package is the recent work of Garg et al~\cite{DBLP:journals/iacr/Garg13} that shows how to realize functional encryption. This notion is tightly connected to the {\em obfuscation} of programs. Roughly speaking, obfuscation of a program means that the program preserves its functionality while becoming unintelligible to the outside. It was known for more than 10 years that so-called black-box obfuscation is  impossible~\cite{DBLP:journals/eccc/ECCC-TR01-057}, however a weaker notion called indistinguishable obfuscation is possible for any polynomial-size circuit.

Surprisingly, this weaker notion is already sufficient to achieve {\em functional encryption}, i.e. to allow a key holder to learn a well-specified function of the encrypted data but nothing else. The construction uses a simplification of so-called multi-linear maps~\cite{DBLP:conf/eurocrypt/GargGH13}.
The work of Garg et al.~immediately raises some open questions that we would like to address.
\begin{itemize}
\item Exploit the power of {\em indistinguishable obfuscation}: It comes quite at a surprise that indistinguishable obfuscation is actually enough to realize functional encryption. Given this, what else can be achieved with this notion?
\item Weaken the complexity assumptions: The construction of {\em indistinguishable obfuscation} requires a heuristic complexity assumption. The question is whether this notion can be achieved by a cryptographic standard assumption.
\item Improve on Efficiency: Our goal in the project is to realize functionalities that are interesting to compute on big data (e.g. computing statistics). Which functionalities can be realized especially fast. Can we improve on efficiency by restricting on a limited class of functionalities. How fast is the realization when implemented?
\end{itemize}



\subsubsection{Work Package B.2: Enhance cryptographic certification properties}
If the cryptographic realizations of other Work Packages of B.1 are applied in a large scale for big data applications, it is mandatory that users can rely on a guarantee that their data's integrity, confidentiality and privacy are protected. Of course, cryptographic security proofs guarantee this for any efficient adversary, albeit {\em only if the schemes are implemented in a way as originally specified}. E.g. almost all cryptographic heavily rely on (pseudo)randomness in a very strong sense. Namely, the security guarantees might completely collapse if the used bits provide for instance some bias.

Thus in order to guarantee a wide-spread acceptance of cryptography for highly sensitive big data, like medical information, it is mandatory to define ``proofs'' that show that the cryptography was used in a sense as specified. We will proceed in the formalization of these proofs/certification as follows.
\begin{itemize}
\item Define a way that formalizes the notion of proof/certification of a cryptographic scheme. Can this only be defined interactively, i.e. does it require a protocol between a prover and a verifier? If yes, how many rounds of interaction do we need in such protocol? If no, how short are the proofs and how efficiently can they be generated and checked?
\item Which functionalities of a cryptographic scheme can easily be certified? Which ones are hard/impossible to certify? Notice that there must be impossibility results, since certification in its strongest sense provides an equivalence test for two Turing machines, a problem that is known to be undecidable  in general. On the positive side, we already showed in our preliminary work that interesting properties can be certified efficiently.
\item Which cryptographic schemes allow for an easy embedding of backdoors that enable somebody to decrypt without a secret key? Is there a general tranformation that embeds a trapdoor into a scheme in an oblivious manner? Or is it possible to construct {\em backdoor-free cryptography} in the sense that any scheme that includes a backdoor can be easily detected. 	
\end{itemize}

\subsubsection{Work Package C.1: Cryptography in the streaming model}

Building on the work in Packages A.1, B.1, and B.2 we plan to study which functionalities can be achieved within some approximation ratio 
in sublinear time in the encrypted domain. Notice that we are not allowed to decrypted in order to apply standard 
non-cryptographic solutions to the problem. We would like to stress that such a setting has not been studied in the cryptographic community so far. Hence we have to start from scratch and will proceed in the following way:
\begin{itemize}
\item Define a proper cryptographic model that serves as an analogue of the classical streaming model. Study and adapt classical security notions such as indistinguishability of encryptions under CPA/CCA-attacks in the new model. A cryptographic scheme by definition always satisfies two properties, its functionality and the security in a well-defined setting (for any polynomial time adversary). Here, we first start with the security property. We take related notions like on-line encryption \cite{online} into account.
\item Which of the interesting functionalities of classical streaming algorithms like e.g. heavy hitters, frequency moments or entropy can be transferred to algorithms in the encrypted domain? How does the performance suffer from restricting on encrypted data?
\item Which of the efficiently realizable functionalities from Project B.1 transfers to the streaming world? How does the approximation guarantee suffer from the streaming model? Classify which functionalities in the encrypted domain are especially suited towards streaming.
\end{itemize}

\subsection{Data handling}

The results of the theoretical work is planned to be published at conferences, workshops, and in journals. This ensures
dissemination and availability of the project's results. In addition, we will use the usual electronic archives
like IACR's \texttt{eprint} service, and the PI's home pages to make full versions available in a lasting way.
As for experimental data, like software, we will also make these public, as part of the corresponding publication,
and through electronic archives.


\subsection{Other information}

 %Please use this section for any additional information you feel is relevant which has not been provided elsewhere.
Not applicable.

\subsection{Descriptions of proposed investigations involving experiments on humans, human materials or animals}

Not applicable.

\subsection{Information on scientific and financial involvement of international cooperation partners}

Not applicable.


%%%%%%%%%%%%%%%%%%%%%%%%%%%%%%%%%%%%%%%%%%%%%%%%%%%%%%%%%%%%%%%%%%%

\section{Bibliography}


\begingroup
\renewcommand{\section}[2]{}% eliminatinf reference section title / dirty hack

%\begin{thebibliography}{ABCD}

%\item[] \hspace{-0.6cm} Own work is marked with *.

%\reversemarginpar
%\marginparwidth 1pt

%\def\shortbib{0}
\bibliographystyle{plain}
\bibliography{lit,local_bib}

\iffalse
\bibitem[ABMRS13]{dedup2}
Martin Abadi, Dan Boneh, Ilya Mironov, Ananth Raghunathan and Gil Segev:
Message-Locked Encryption for Lock-Dependent Messages. CRYPTO 2013, LNCS, Springer, 2013.

\bibitem[ABCHSW12]{P2:ABCHSW12} J. H. Ahn, D. Boneh, J. Camenisch, S. Hohenberger, A. Shelat, and B. Waters. Computing
on Authenticated Data. In: TCC 2012. Ed. by R. Cramer. Vol. 7194. Lecture Notes in
Computer Science. Taormina, Sicily, Italy: Springer, Berlin, Germany, 2012, pp. 1-20.

\bibitem[BMS13]{P2:Backes} M. Backes, S. Meiser, and D. Schr\"oder. Highly Controlled, Fine-grained Delegation of Signing
Capabilities. IACR Cryptology ePrint Archive, 408/2013. 2013.

\bibitem[BGIRSV01]{DBLP:journals/eccc/ECCC-TR01-057}
B.~Barak, O.~Goldreich, R.~Impagliazzo, S.~Rudich, A.~Sahai, S.~P. Vadhan, and
  K.~Yang.
\newblock On the (im)possibility of obfuscating programs. 
\newblock CRYPTO 2001, LNCS, Springer. Full version: {\em Electronic Colloquium on Computational Complexity (ECCC)},
  8(057), 2001.

\bibitem[BBN07]{BBN07}
Mihir Bellare, Alexandra Boldyreva, Adam O'Neill: Deterministic and Efficiently Searchable Encryption. CRYPTO 2007, LNCS, Springer, 
pp.~535-552, 2007.

\bibitem[BKR13]{dedup}
Mihir Bellare, Sriram Keelveedhi, Thomas Ristenpart: Message-Locked Encryption and Secure Deduplication. EUROCRYPT, LNCS, Springer, 
pp.~296-312, 2013.

\bibitem[BN02]{P2AC:BelNev02} M. Bellare and G. Neven. Transitive Signatures Based on Factoring and RSA. In: Advances
in Cryptology - ASIACRYPT 2002. Ed. by Y. Zheng. Vol. 2501. Lecture Notes in
Computer Science. Queenstown, New Zealand: Springer, Berlin, Germany, 2002, pp. 397-414.



\bibitem[BPMO12]{BPMO12} Erik-Oliver Blass, Roberto Di Pietro, Refik Molva, Melek \"Onen: PRISM - Privacy-Preserving Search in MapReduce. Privacy Enhancing Technologies 2012, LNCS 7384, Springer, pp.~180-200, 2012.

\bibitem[BSN08]{BSN08} Alexandra Boldyreva, Serge Fehr, Adam O'Neill: On Notions of Security for Deterministic Encryption, and Efficient Constructions without Random Oracles. CRYPTO 2008, LNCS, Springer, pp.~335-359, 2008.

\bibitem[BT04]{online} 
Alexandra Boldyreva, Nut Taesombut: Online Encryption Schemes: New Security Notions and Constructions. CT-RSA 2004, LNCS, Springer, 
pp.~1-14, 2004.

\bibitem[BGLS03]{BGLS03}
D. Boneh, C. Gentry, H. Shacham, and B. Lynn:
Aggregate and Verifiably Encrypted Signatures from Bilinear Maps.
In proceedings of Eurocrypt 2003, LNCS 2656, Springer, pp. 416-432, 2003. 

\bibitem[BSW06]{DBLP:conf/eurocrypt/BonehSW06}
D.~Boneh, A.~Sahai, and B.~Waters.
\newblock Fully collusion resistant traitor tracing with short ciphertexts and
  private keys.
\newblock In S.~Vaudenay, editor, {\em EUROCRYPT}, volume 4004 of {\em Lecture
  Notes in Computer Science}, pages 573--592. Springer, 2006.

\bibitem[BSW11]{DBLP:conf/tcc/BonehSW11}
D.~Boneh, A.~Sahai, and B.~Waters.
\newblock Functional encryption: Definitions and challenges.
\newblock In Y.~Ishai, editor, {\em TCC}, volume 6597 of {\em Lecture Notes in
  Computer Science}, pages 253--273. Springer, 2011.

\bibitem[BF11]{P2EC:BonFre11} D. Boneh and D. M. Freeman. Homomorphic Signatures for Polynomial Functions. In:
Advances in Cryptology - EUROCRYPT 2011. Ed. by K. G. Paterson. Vol. 6632. Lecture
Notes in Computer Science. Tallinn, Estonia: Springer, Berlin, Germany, 2011, pp. 149-168.

\bibitem[BGI13]{P2:Boyle}
E. Boyle, S. Goldwasser, and I. Ivan. Functional Signatures and Pseudorandom Functions.
IACR Cryptology ePrint Archive, 401/2013. 2013.


\bibitem[CLX09]{P2RSA:ChaLimXu09}
E.-C. Chang, C. L. Lim, and J. Xu. Short Redactable Signatures Using Random Trees. In: CT-RSA 2009. Ed. by M. Fischlin. Vol. 5473. Lecture Notes in Computer Science.
San Francisco, CA, USA: Springer, Berlin, Germany, 2009, pp. 133-147.

\bibitem[CKLYBNO06]{ml} Cheng-Tao Chu, Sang Kyun Kim, Yi-An Lin, YuanYuan Yu, Gary R. Bradski, Andrew Y. Ng, Kunle Olukotun: Map-Reduce for Machine Learning on Multicore. NIPS, pp.~281-288, 2006.

\bibitem[DG04]{OSDI04} Jeffrey Dean and Sanjay Ghemawat:
 MapReduce: Simplified Data Processing on Large Clusters.
 OSDI'04: Sixth Symposium on Operating System Design and Implementation,
San Francisco, CA, December, 2004. 

\bibitem[D93]{P2Desmedt:1993:CSR:283751.283834} Y. Desmedt. Computer security by redefining what a computer is. In: Proceedings on the 1992-
1993 workshop on New security paradigms. NSPW 92-93. ACM, 1993, pp. 160-166.

\bibitem[EOM13]{EOM13} Kaoutar Elkhiyaoui, Melek \"Onen, and Refik Molva:
Privacy preserving delegated word-search in the cloud,
TCLOUDS 2013, Workshop on Trustworthy Clouds, in connection with ESORICS, 2013.

\bibitem[F12]{P2PKC:Freeman12} D. M. Freeman. Improved Security for Linearly Homomorphic Signatures: A Generic Framework.
In: PKC 2012: 15th International Workshop on Theory and Practice in Public Key Cryptography.
Ed. by M. Fischlin, J. Buchmann, and M. Manulis. Vol. 7293. Lecture Notes in Computer Science.
Darmstadt, Germany: Springer, Berlin, Germany, 2012, pp. 697-714.

\bibitem[CFW12]{P2PKC:CatFioWar12} D. Catalano, D. Fiore, and B. Warinschi. Efficient Network Coding Signatures in the Standard
Model. In: PKC 2012: 15th International Workshop on Theory and Practice in Public Key Cryptography.
Ed. by M. Fischlin, J. Buchmann, and M. Manulis. Vol. 7293. Lecture Notes in Computer
Science. Darmstadt, Germany: Springer, Berlin, Germany, 2012, pp. 680-696.

\bibitem[GGH13]{DBLP:conf/eurocrypt/GargGH13}
S.~Garg, C.~Gentry, and S.~Halevi.
\newblock Candidate multilinear maps from ideal lattices.
\newblock In T.~Johansson and P.~Q. Nguyen, editors, {\em EUROCRYPT}, volume
  7881 of {\em Lecture Notes in Computer Science}, pages 1--17. Springer, 2013.

\bibitem[GGHRSW13]{DBLP:journals/iacr/Garg13}
S.~Garg, C.~Gentry, S.~Halevi, M.~Raykova, A.~Sahai, and B.~Waters.
\newblock Attribute-based encryption for circuits from multilinear maps.
\newblock {\em FOCS}, 2013.

\bibitem[GKPVZ13a]{DBLP:conf/crypto/GoldwasserKPVZ13}
S.~Goldwasser, Y.~T. Kalai, R.~A. Popa, V.~Vaikuntanathan, and N.~Zeldovich.
\newblock How to run turing machines on encrypted data.
\newblock In R.~Canetti and J.~A. Garay, editors, {\em CRYPTO (2)}, volume 8043
  of {\em Lecture Notes in Computer Science}, pages 536--553. Springer, 2013.

\bibitem[GKPVZ13b]{DBLP:conf/stoc/GoldwasserKPVZ13}
S.~Goldwasser, Y.~T. Kalai, R.~A. Popa, V.~Vaikuntanathan, and N.~Zeldovich.
\newblock Reusable garbled circuits and succinct functional encryption.
\newblock In D.~Boneh, T.~Roughgarden, and J.~Feigenbaum, editors, {\em STOC},
  pages 555--564. ACM, 2013.



%\bibitem{AG} Sanjeev Arora, Rong Ge, ``Learning Parities with Structured Noise'', Electronic Colloquium on Computational Complexity (ECCC) 17: 66 (2010)

\bibitem[JMSW02]{P2RSA:JMXW02}
R. Johnson, D. Molnar, D. X. Song, and D. Wagner. Homomorphic Signature Schemes. In: Topics
in Cryptology - CT-RSA 2002. Ed. by B. Preneel. Vol. 2271. Lecture Notes in Computer Science.
San Jose, CA, USA: Springer, Berlin, Germany, 2002, pp. 244-262.

\bibitem[KR11]{KR11} Seny Kamara, Mariana Raykova: Parallel Homomorphic Encryption. IACR Cryptology ePrint Archive 2011, No.~596, 2011.
See also WAHC'13 - Workshop on Applied Homomorphic Cryptography, Associated with Financial Crypto \& Data Security 2013.

\bibitem[KSV10]{KSV10} H. Karloff, S. Suri, and S. Vassilvitskii. A model of computation for mapreduce. In Symposium on
Discrete Algorithms (SODA '10), pages 938--948. SIAM, 2010.

\bibitem[LJLC12]{LJLC12}
Jingwei Li, Chunfu Jia, Jin Li, Xiaofeng Chen: Outsourcing Encryption of Attribute-Based Encryption with MapReduce. ICICS 2012,
LNCS 7618, Springer, pp.~191-201, 2012.

\bibitem[LD10]{LD10} J. Lin and C. Dyer. Data-Intensive Text Processing with MapReduce. Morgan \& Claypool, 2010.


\bibitem[LMRS04]{LMRS04} 
Anna Lysyanskaya,
Silvio Micali,
Leonid Reyzin,
Hovav Shacham.
Sequential Aggregate Signatures from Trapdoor Permutations.
In EUROCRYPT 2004, LNCS 3027, Springer, pp.~74-90, 2004.


\bibitem[MR02]{P2RSA:MicRiv02b}
S. Micali and R. L. Rivest. Transitive Signature Schemes. In: CT-RSA 2002.
Ed. by B. Preneel. Vol. 2271. Lecture Notes in Computer Science. San Jose, CA, USA: Springer,
Berlin, Germany, 2002, pp. 236-243.


\bibitem[PHGR12]{DBLP:conf/sp/ParnoHG013}
B.~Parno, J.~Howell, C.~Gentry, and M.~Raykova.
\newblock Pinocchio: Nearly practical verifiable computation.
\newblock In {\em IEEE Symposium on Security and Privacy}, pages 238--252. IEEE
  Computer Society, 2013.


\bibitem[SW05]{DBLP:conf/eurocrypt/SahaiW05}
A.~Sahai and B.~Waters.
\newblock Fuzzy identity-based encryption.
\newblock In R.~Cramer, editor, {\em EUROCRYPT}, volume 3494 of {\em Lecture
  Notes in Computer Science}, pages 457--473. Springer, 2005.


\bibitem[T10]{T10} Javier Tordable: MapReduce for Integer Factorization. CoRR abs/1001.0421, 2010.


%[13] M. T. Goodrich and M. Mitzenmacher. Mapreduce parallel
%cuckoo hashing and oblivious ram simulation. CoRR,
%abs/1007.1259, 2010.

\fi

%\end{thebibliography}

\endgroup


%%%%%%%%%%%%%%%%%%%%%%%%%%%%%%%%%%%%%%%%%%%%%%%%%%%%%%%%%%%%%%%%%%%


\section{Requested modules/funds}
%Explain each item for each applicant (stating last name, first name).

\subsection{Basic Module}

\subsubsection{Funding for Staff}

We apply for the following funding for staff, following the DFG's personal rates for 2013 (DFG-Vordruck 60.12). 
Due to competitive nature of positions in Computer Science and in the area of IT security, we request
funding for full doctoral student positions (100\%). We apply for funding of two doctoral
students for the entire duration of the project. One student will be associated to TU Darmstadt (TUD)
and work on packages A.1, A.2, and C.1, the other student will be associated to Ruhr-University Bochum (RUB)
and work on packages B.1, B.2, and C.1. Student Assistents (one for each project partner) have been calculated with 10h/months
for the full duration of the project, with average costs of 11 EUR/h. They should support the implementations
in the corresponding work packages.

\quad % make more space

\noindent {\footnotesize
\begin{tabular}{|l||l|l|l||r|r|r|}
 \hline No &  Type & assoc.~to & Description & Y1 & Y2 & Y3\\
 \hline \hline
 1 & Doctoral Student, 100\%  & TUD & WP A.1,A.2,C.1 & 57,300\euro & 57,300\euro & 57,300\euro \\
 2 & Doctoral Student, 100\%  & RUB& WP B.1,B.2,C.1 & 57,300\euro & 57,300\euro & 57,300\euro\\
 3 & Student Assistent, 100\% & TUD & Implementations & 1,320\euro & 1,320\euro & 1,320\euro\\
 4 & Student Assistent, 100\% & RUB & Implementations & 1,320\euro & 1,320\euro & 1,320\euro \\
 \hline
  & Total Amount  & & & 117,240\euro & 117,240\euro & 117,240\euro\\ \hline 
\end{tabular}
}

\quad

%The doctoral students should ...

%{\em two TVL 13-position} and 
%{\em two studentische Hilfskraft position} (10 h per week).
%We apply for a 3-year funding period of all positions, starting in June 2014.\\


\subsubsection{Direct Project Costs}


\paragraph{Equipment up to EUR 10,000, Software and Consumables.}
We request an overall funding of 3,000 EUR for each applicant over the entire project duration, 6,000 EUR in total,
to carry out experiments in Amazon's commercial Elastic MapReduce. Using Amazon's platform instead of local
computing centers of both universities allows for an easier benchmarking and a smoother collaboration.

\paragraph{Travel Expenses.}
We apply for 8,000 Euro travel funding per year, i.e. for a total travel funding of 24,000 Euro.
This funding will cover expenses of the project partners for traveling between Darmstadt and Bochum
(500 EUR per year per partner), as well as attendances of potential  meetings of the SPP (500 EUR per year per 
partner). In addition, we calculate with 3000 EUR per year and partner for presenting the project results
at internationally renowned conferences and workshops (where we assume 1,500 EUR for an intercontinental conference, 
1,000 EUR for a European conference, and 500 EUR for a workshop attendance).


\iffalse
\paragraph{Visiting Researchers (excluding Mercator Fellows)}

\paragraph{Expenses for Laboratory Animals}


\paragraph{Other Costs}

\paragraph{Project-related publication expenses}


\subsection{Instrumentation}

\paragraph{Equipment exceeding Euro 10,000}


\paragraph{Major Instrumentation exceeding Euro 50,000}


\subsection{Module Temporary Position for Funding}


\subsection{Module Replacement Funding}
\fi

\iffalse %nicht relevant
4.4	Module Temporary Clinician Substitute


[Text]
4.5	Module Mercator Fellows

[Text]

4.6	Module Workshop Funding

[Text]

4.7	Module Public Relations Funding

[Text]

\fi

%%%%%%%%%%%%%%%%%%%%%%%%%%%%%%%%%%%%%%%%%%%%

\section{Project requirements}

\subsection{Employment status information}

\noindent
{\bf Marc Fischlin}, Dr.rer. nat \\
W3-Professor\\
%Postdoc TU Darmstadt (tenured, on sabbatical for Heisenberg Professorship)\\
Year of birth: 1973, Nationality: German\\[0.3cm]
Cryptography and Complexity Theory\\
Department of Computer Science\\
Technische Universit\"at Darmstadt\\
Karolinenplatz 5, 64289 Darmstadt, Germany\\[0.3ex]
Phone office: +49-(0)6151/16-25730\\
Fax office: +49-(0)6151-16-22487\\
E-Mail: marc.fischlin@cryptoplexity.de \\[0.3cm]
Privat address: Charlotte-Posenenske-Str.58, 65197 Wiesbaden\\
Phone private: +49-(0)611/9882730\\ \\



\noindent {\bf Alexander May}, Dr. rer. nat\\
W3-Professor (tenured)\\
Year of birth: 1974, Nationality German \\[0.3cm]
Faculty of Mathematics, NA 5/73\\
Ruhr-University Bochum\\
Universit\"atsstr. 150, 44801 Bochum\\[0.3cm]
Phone office: 0234/32-23261\\
Fax office: 0234/32-14430 \\
E-Mail: alex.may@rub.de \\[0.3cm]
Privat address: Im Haarmannsbusch 34, 44797 Bochum\\
Phone private: 015159-207944\\
%

% For each applicant, state the last name, first name, and employment status (including duration of contract and funding body, if on a fixed-term contract).

%[Text]  

\subsection{First-time proposal data}
%Only if applicable: Last name, first name of first-time applicant.
\vspace*{-1.5ex}
Not applicable.
%[Text]

\subsection{Composition of the project group}

%List only those individuals who will work on the project but will not be paid out of the project funds. State each person’s name, academic title, employment status, and type of funding.
\textbf{TU Darmstadt:}
\begin{itemize}
\item Prof. Dr. Marc Fischlin, Chair for Cryptography and Complexity Theory %(funded through Heisenberg-Program FI 940/3-1 of DFG) 
%\item Dr. Pooya Farshim (funded through Heisenberg grant FI 940-1/1)
%\item Dipl.-Inform. Paul Baecher (funded through Heisenberg grant FI 940-1/1)
\item M.Sc.(Math) Jacqueline Brendel (funded through DFG Doctoral College 2050 Privacy and Trust for Mobile Users)
\item M.Sc.(Math) Victoria Fehr (funded through TU Darmstadt)
\item M.Sc.(Math) Tommaso Gagliardoni (funded through BMBF/Hesse Security Competence Center CRISP) 
\item M.Sc.(CS), M.Sc.(CS) Felix G\"unther (funded through DFG Collaborative Research Center 1119 CROSSING)
\item M.Sc.(Math) Christian Janson (funded through TU Darmstadt) 
\item M.Sc.(Math) Giorgia Azzurra Marson  (funded through DFG Collaborative Research Center 1119 CROSSING)
\item M.Sc.(CS) Sogol Mazahero (funded through TU Darmstadt)
%\item M.Sc.(CS) Arno Mittelbach (funded through State Hesse, LOEWE center CASED) 
\end{itemize}
Support: 1 secretary.

\quad

\noindent
{\bf Ruhr-University Bochum:}
\begin{itemize}
\item Prof. Dr. Alexander May, chair of the group Crypto \& IT security
\item Emmy Noether group leader Dr. Christopher Wolf (associated)
\item Dipl.-Math. Gottfried Herold 
\item Dipl.-Ing. Ilya Ozerov (funded by SPP 1307 -- Algorithm Engineering) 
\item Dipl.-Math. Elena Kirshanova (funded by GRK 1817 -- Ubiquitous Cryptography)
\item Dipl.-Ing. Stefan Hoffman (third party funding) 
\end{itemize}
Support: 1 secretary and 1 technical assistant.



\subsection{Cooperation with other researchers}

\subsubsection*{Researchers with whom you have agreed to cooperate on this project}

%We plan to continue our cooperations with the following colleagues:

\noindent
\textbf{Marc Fischlin:}
%
\begin{itemize}
\item Prof.~Kenny Paterson (RHUL, UK)
\item Prof.~Douglas Stebila (McMaster University, Canada)
\item Prof.~Bogdan Warinschi (U Bristol, UK)
\end{itemize}

\noindent
\textbf{Alexander May:}
%
\begin{itemize}
\item Prof. Dr. Antoine Joux (University Versailles), G\"odel prize 2013, visis our group in Nov 2013 for a research stay
\item Prof. Dr. Johannes Bl\"omer (Paderborn University)
\item Prof. Dr. Christian Sohler (TU Dortmund), joint seminar ```Perlen der Theoretischen Informatik''' since 2009
\item Prof. Dr. Hans Simon (Ruhr-University Bochum)
\end{itemize}
%


\subsubsection*{Researchers with whom you have collaborated scientifically within the past three years}

\textbf{Marc Fischlin:}
\begin{itemize}
\item Prof.~Michael Backes (U Saarland, Germany)
\item Dr.~David Bernhard, Prof.~Bogdan Warinschi (U Bristol, UK)
\item Dr.~Jean Paul Degabriele, Prof.~Kenny Paterson (RHUL, UK)
\item Prof.~Amir Herzberg (Tel-Aviv University)
\item Dr.~Anja Lehmann (IBM Zurich, Switzerland)
\item Prof.~Krzysztof Pietrzak (IST, Austria)
\item Dr.~Benedikt Schmidt (IMDEA, Madrid, Spain)
\item Prof.~Dominique Schr\"oder (U N\"urnberg-Erlangen, Germany)
\end{itemize}

\noindent
\textbf{Alexander May:}
\begin{itemize}
\item Professors Eike Kiltz and Hans Simon (Ruhr-University Bochum)
\item Professor Antoine Joux (University Versailles)
\item Professor Johannes Bl\"omer (Paderborn University)
\item Professor Nigel Smart (University of Bristol, UK)
\end{itemize}


\subsection{Scientific equipment}

Besides standard equipment the project partners 
\textbf{ich brauche nix besonderes!!!!}


\subsection{Project-relevant interests in commercial enterprises}

Not applicable.

%%%%%%%%%%%%%%%%%%%%%%%%%%%%%%%%%%%%%%%%%%
\section{Additional information}

%If applicable, please list proposals requesting major instrumentation and/or those previously submitted to a third party here.


We have not requested funding for this project from any other sources. In the event that we submit such a request, we will inform the Deutsche Forschungsgemeinschaft immediately.

The DFG liaison officer's of TU Darmstadt and Ruhr-University Bochum will be informed of this research funding request.



%\newpage

\iffalse
%\newpage
\section{Prerequisites for carrying out the project}

For the project we do not have other funding sources. TU Darmstadt and Ruhr-University Bochum will provide all other required facilities (office, computing resources, etc.)

\subsection{Our team -- Organisational structure of our groups}
%
{\bf Ruhr-University Bochum:}
\begin{itemize}
\item Prof. Dr. Alexander May, chair of the group Crypto \& IT security
\item Emmy Noether group leader Dr. Christopher Wolf (associated)
\item Dipl.-Math. Gottfried Herold 
\item Dipl.-Ing. Ilya Ozerov (currently financed by SPP 1307 -- Algorithm Engineering) 
\item Dipl.-Math. Elena Kirshanova (currently financed by GRK 1817 -- Ubiquitous Cryptography)
\item Dipl.-Ing. Stefan Hoffman (third party funding) 
\end{itemize}

Support: 1 secretary and 1 technical assistant

\subsection{Cooperation with other scientists}

We plan to continue our cooperations with the following colleagues.
%
\begin{itemize}
\item Prof. Dr. Antoine Joux (University Versailles), G�del prize 2013, visis our group in Nov 2013 for a research stay
\item Prof. Dr. Johannes Bl�mer (Paderborn University)
\item Prof. Dr. Christian Sohler (TU Dortmund), joint seminar ```Perlen der Theoretischen Informatik''' since 2009
\item Prof. Dr. Hans Simon (Ruhr-University Bochum)
\end{itemize}
%

\subsection{Scientific equipment}

fully existing

\subsection{Running costs for material}

not necessary

\subsection{Other requirements}

none


%%%%%%%%%%%%%%%%%%%%%%%%%%%%%%%%%%%%%%%%%%%%%%%%%%%%%%%%%%%%%%%%%%%%%%

%\newpage
\section{Declarations}


We have not requested funding for this project from any other sources. In the event that we
submit such a request, we will inform the Deutsche Forschungsgemeinschaft immediately.

The DFG liaison officer's of TU Darmstadt's and Ruhr-University Bochum will be informed of this research funding request.


%%%%%%%%%%%%%%%%%%%%%%%%%%%%%%%%%%%%%%%%%%%%%%%%%%%%%%%%%%%%%%%%%%%%%%

\section{Signature}

\vspace{2cm}

TU Darmstadt, 30.09.2013 \hspace{4cm} (Marc Fischlin) \hspace{4cm} Bochum, 15.03.2011 \hspace{4cm} (Alexander May) \\


%%%%%%%%%%%%%%%%%%%%%%%%%%%%%%%%%%%%%%%%%%%%%%%%%%%%%%%%%%%%%%%%%%%%%%

\section{Attachments}

\begin{itemize}
\item CV of the applicants
\item Paper: A. May, I. Ozerov, ``On Merging Lists Consistently: Solving Subset Sum in $2^{0.287n}$'', currently in submission.
\end{itemize}

%%%%%%%%%%%%%%%%%%%%%%%%%%%%%%%%%%%%%%%%%%%%%%%%%%%%%%%%%%%%%%%%%%%%%%

\newpage
\subsection{Prof. Dr. rer. nat. Alexander May}

born on March 22, 1974 in Friedberg (Germany)\\
Ruhr-University Bochum\\
Chair for Cryptology and IT-Security\\
Faculty of Mathematics\\
Phone: +49 234 32 23261\\
Email: alex.may@rub.de\\

\subsubsection*{Education \& Professional Experience}
\begin{tabular}{ll}
{10.93 -- 07.99} & Computer Science study, J.W. Goethe-Universit\"at in Frankfurt/Main \\
{09.99 -- 02.00} & PhD, Computer Science Department, ETH Z\"urich \\
{03.00 -- 12.03} & PhD, Computer Science Department, Paderborn University \\
{01.04 -- 09.05} & Post-Doc in DFG-Priority Programme \\
%
& {``Sicherheit in der Informations- und
Kommunikationstechnik''}\\
%
{10.05 -- 09.07} & Juniorprofessor, Cryptographic Protocols, TU Darmstadt \\
{since 10.07} & W3-Professor, Cryptology and IT-Security, Ruhr-University Bochum \\
\end{tabular}




\vskip 0.3cm



%\subsubsection*{Academic Honors and Awards}

%\begin{tabular}{ll}
%
%2009 & Best Paper Award PKC together with Maike Ritzenhofen for the paper \\
%& \textit{``Implicit Factoring: On Polynomial Time Factoring Given Only an Implicit Hint''} \\
%2006 & Best Paper Award PKC together with Daniel Bleichenbacher for the paper \\
%&  {\it ``New Attacks on RSA with Small Secret CRT-Exponents''} \\
%2005 & Price of Faculty Electrical Engineering, Computer Science and Mathematics\\
%& for the PhD thesis  {\it ``New RSA Vulnerabilities Using Lattice Reduction Methods''} \\
%& \\
%2007 & Price of Student Body Computer Science at TU Darmstadt for the best lecture\\
%2006 & Price of Student Body Computer Science at TU Darmstadt for the best lecture\\
%2004 & Weierstra\ss-Price of Faculty Electrical Engineering, Computer Science and Math.\\
%&  at Paderborn University for excellent teaching\\
%\end{tabular}


\subsubsection*{Research Grants Relevant for the Application}

\begin{itemize}
\item DFG-GRK 1817 Ubiquitous Cryptography, ``Outsourced Computation'', since 2013
\item ERC Starting Grant 307952 (Partner), ``Fast and Sound Cryptography'', since 2012 
\item DFG-SPP 1307 Algorithm Engineering, ``Algorithms for Subset Sum, Lattices and Linear Codes'', 2011-13
\item DFG RUB-RS (ExIni), ``Coding-Based Cryptanalysis'', 2009-12
\item DFG, ``Lattice-Based Solving of Polynomial Equations'', 2007-10
\item EU, ECRYPT II -- European Network of Excellence in Cryptology, since 2008
\item DFG-SPP 1079 IT Security, ``Lattice Attacks on RSA'', 2003-2005
\end{itemize}

\subsection*{Brief Description of the Chair for Cryptology and IT-Security}
The research group interests cover all aspects of modern cryptology, algorithmic number theory and discrete mathematics.
The chair's special focus lies on algorithmic aspects of cryptography.
Our spectrum of cryptographic applications covers problems from factoring, discrete logarithms, lattice and coding theory, diophantine equations and quantum algorithms.


\vskip 0.3cm


\subsubsection*{Five Publications most Relevant for the Application}
%\footnote{Die wichtigsten Beitr\"age sind \textbf{fett} gedruckt.}}

\vspace{0.0cm}


%\textbf{Book chapter  (reviewed)}

%\begin{itemize}
%\item \textit{``Using LLL-Reduction for Solving RSA and Factorization Problems: A Survey''} \\
%In ``The LLL Algorithm --Survey and Applications'', Editors Phong Nguyen and Brigitte Vall\'ee, Springer, 2010.
%\end{itemize}



%\noindent \textbf{Journals (reviewed)}

%\begin{itemize}
%\item \textit{``Deterministic Polynomial Time Equivalence of Computing
%the RSA Secret Key and Factoring''}  with Jean-S\'ebastien Coron \\
%Journal of Cryptology, 2007.

%\item \textit{``Perspectives for Cryptographic Long-Term Security''} with Johannes Buchmann and Ulrich Vollmer \\
%Communications of the ACM, Vol. 94(9), 2006.

%\item \textit{``Krypto 2020''} with Johannes Buchmann, Erik Dahmen and Ulrich Vollmer \\
%KES -- The Information Security Journal, Vol. 5, 2005.

%\end{itemize}

%\noindent \textbf{Conference Papers (reviewed)}

%\vspace{0.8cm}

\begin{itemize}
\item Saqib Kakvi, Eike Kiltz, Alexander May, \textit{``Certifying RSA''},  In Advances in Cryptology (Asiacrypt 2012), Lecture Notes in Computer Science, Springer-Verlag, 2012.
\item Anja Becker, Antoine Joux, Alexander May, Alexander Meurer, 
	\textit{``Decoding Random Binary Linear Codes in $2^{n/20}$: How 1+1=0 Improves Information Set Decoding''} , In Advances in Cryptology (Eurocrypt 2012), Lecture Notes in Computer Science, Springer-Verlag, 2012.
\item Alexander May, Alexander Meurer, Enrico Thomae
	\textit{``Decoding Random Linear Codes in $O(2^{0.054n})$''} , In Advances in Cryptology (Asiacrypt 2011), Lecture Notes in Computer Science, Springer-Verlag, 2011.
\item Jake Loftus, Alexander May, Nigel P. Smart, Frederik Vercauteren
	\textit{``On CCA-Secure Somewhat Homomorphic Encryption''} , In Selected Areas in Cryptography (SAC 2011), Lecture Notes in Computer Science, Springer-Verlag, 2011.
\item Wilko Henecka, Alexander May, Alexander Meurer
	\textit{``Correcting Errors in RSA Private Keys''}, In Advances in Cryptology (Crypto 2010), Lecture Notes in Computer Science, Springer-Verlag, 2010.
\end{itemize}

\fi

\end{document}


\iffalse DFG Vorlage

Project Description - Project Proposals


[First name  last name, city of all applicants] 
 





2	Objectives and work programme
2.1	Anticipated total duration of the project

[Text]
2.2	Objectives

[Text]
2.3	Work programme incl. proposed research methods
For each applicant

[Text]
2.4	Data handling

[Text]

2.5	Other information
Please use this section for any additional information you feel is relevant which has not been provided elsewhere.

[Text]

2.6	Descriptions of proposed investigations involving experiments on humans, human materials or animals

[Text]

2.7	Information on scientific and financial involvement of international cooperation partners

[Text]



3	Bibliography

[Text]




\fi


\iffalse
%===============
I Priority Programme
For individual project proposals within an established Priority Programme, note that the fund-ing duration (part A of the proposal) and the funding periods are specified in the call for pro-posals.
For the project description (part B of the proposal) note the following:
Each proposal must be accompanied by a description of how the project is integral to the Priority Programme, both in terms of subject matter and organisation. This includes a de-scription of the cooperation with others participating within the Priority Programme. The en-visaged realisation of the project in cooperation with other applicants may be demonstrated in particular by the joint training of early career researchers, or the use of methods by multi-ple projects as part of a network.
DFG form 54.01 - 04/13 page 11 of 14
Deutsche Forschungsgemeinschaft
Kennedyallee 40 ∙ 53175 Bonn ∙ postal address: 53170 Bonn
phone: + 49 228 885-1 ∙ fax: + 49 228 885-2777 ∙ postmaster@dfg.de ∙ www.dfg.de DFG
All applicants involved in submitting a proposal within an established Priority Programme are obliged to promptly provide the overall coordinator with all of the information necessary for drawing up the interim reports and the final report for the Priority Programme.
\fi