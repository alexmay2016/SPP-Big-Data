%!TEX root = memoc.tex

\subsection{Preliminary Work in the first Project SecOBig}\label{sec:secobig}

In the first phase of the priority program both applicants have conducted a project called \emph{Security-Preserving Operations on Big Data} (\emph{SecOBig}).  The focus of this proposal here has changed now. In the following we report briefly on the achievements of \emph{SecOBig} (so far, with the project still running for about 8 months) and the reason for the shift of topic and a fresh application (instead of a renewal application).

\paragraph{Achievements in SecOBig.}
SecOBig promised to work on efficient operations on secured data, targeted as well as through the deployment of functional encryption and indistinguishable obfuscation, certification of cryptographic primitives, and the invention of new algorithmic techniques for big cryptographic data. At TU Darmstadt, Arno Mittelbach has been working on the project but has meanwhile left academia. At Ruhr-University Bochum, Ilya Ozerov worked on the project, who left academia in February 2016. He was replaced by Elena Kirshanova, who will submit her thesis in November 2016.
%
The following works have been published in the context of the project so far: 

\noindent
Peer-reviewed Publications:
\vspace*{-0.5cm}
\begin{itemize}
 \item   Marc Fischlin, Amir Herzberg, Hod Bin Noon, Haya Shulman:
Obfuscation Combiners. Crypto 2016. This work deals with the certification and obfuscation questions, as it shows how to build robust solutions in light of malicious obfuscators, and reports about implementation results.
%
\item Elena Kirshanova, Alexander May, Friedrich Wiemer: Parallel Implementation of BDD enumeration for LWE, ACNS 2016 ({\em Best Student Paper}). Implements a memory-efficient variant of an enumeration technique for LWE.
%
\item 
Victoria Fehr, Marc Fischlin:
Sanitizable Signcryption: Sanitization over Encrypted Data. In submission, see also IACR cryptographic eprint archive 2015. Provides solutions to allow for controlled modifications of authenticated, encrypted data.
%
\item Alexander May, Ilya Ozerov: On Computing Nearest Neighbors with Applications to Decoding of Binary Linear Codes, Eurocrypt 2015. Proposes a new algorithm for finding correlations within two huge lists in time sub-quadratic in the list length.  
%
\item Rolf Egert, Marc Fischlin, David Gens, Sven Jacob, Matthias Senker, J\"orn Tillmanns: Privately Computing Set-Union and Set-Intersection Cardinality via Bloom Filters. ACISP 2015. This work deals with efficient operations on encrypted data for computing the number of shared elements in large sets. Topic has been inspired by the invited talk of Michael Goodrich about Bloom filters (opening of SPP 1736).
%
\item Gottfried Herold, Elena Kirshanova, and Alexander May. On the Asymptotic Complexity of Solving LWE. IACR Cryptology ePrint Archive 2015, accepted at Journal of Design, Codes and Cryptography. Relates to time, memory and sample complexity of LWE solving algorithms.
%
\item Christina Brzuska, Pooya Farshim, Arno Mittelbach:
Random-Oracle Uninstantiability from Indistinguishability Obfuscation. TCC 2015. Relates to the question of obfuscation, and shows that the random oracle methodology for designing practical solutions may not be applicable in general.
%
\item Elena Kirshanova. Proxy Re-encryption from Lattices.  PKC 2014. Relates to outsourcing computation in a distributed environment without trusted authorities.
%
\item Christina Brzuska, Arno Mittelbach:
Indistinguishability Obfuscation versus Multi-bit Point Obfuscation with Auxiliary Input. Asiacrypt 2014. Relates to the question of obfuscation and technical details about the realizability.
%
\item Alexander May, Ilya Ozerov. A Generic Algorithm for Small Weight Discrete Logarithms in Composite Groups. Selected Areas in Cryptography 2014. Relates to hardness assumption in general cryptographic groups of large size.
%
\item 
Christina Brzuska, Pooya Farshim, Arno Mittelbach:
Indistinguishability Obfuscation and UCEs: The Case of Computationally Unpredictable Sources, Crypto 2014. Relates to the question of obfuscation and shows that obfuscation can actually be used to show negative result.
%
\end{itemize}
%
\vspace*{-0.3cm}
\noindent
Theses:
\vspace*{-0.3cm}
\begin{itemize}
\item Elena Kirshanova. Analysis of Hard Problems on Lattices. Ph.D.~Thesis, Ruhr-University Bochum, November 2016.
%
\item Sven Jacob. Realizing Cryptographic Protocols in the MapReduce-Framework. Ongoing Master Thesis, TU Darmstadt, 2016. Provides cryptographically secure protocols in the MapReduce framework, including implementations in Hadoop and Amazon's Elastic MapReduce (EMR) framework.
%
\item Ilya Ozerov. Combinatorial Algorithms for Subset Sum Problems. Ph.D.~Thesis, Ruhr-University Bochum, February 2016.
%
\item Arno Mittelbach. Random Oracles in the Standard Model --- A Systematic Study of Random Oracle (Un)Instantiability via Universal Computational Extractors and Obfuscation. Ph.D.~Thesis, TU Darmstadt, December 2015.
%
\item Kai Schwierczek. Approximation of the Maximum in Big Data, Master-Thesis TU Darmstadt, 2015. Touches the question how to compute the easy statistics (like the maximum or minimum) on encrypted outsourced data efficiently, by sacrificing precision.
%
\item Leif Both. Cryptography with Streaming Algorithms. Master-Thesis Ruhr-University Bochum 2015. Asks which cryptographic primitives can be realized in a streaming model of computation providing only logarithmic space.

%
\item Tobias Weber. Combiners for Robust Pseudorandom Number Generators, Master-Thesis TU Darmstadt, 2015. Deals with certification in the sense that one builds robust pseudorandom generators in the presence of some malicious generators.
%
\end{itemize}
\vspace*{-0.3cm}
\noindent
Others:
\vspace*{-0.3cm}
\begin{itemize}
\item Dagstuhl seminar about Public-Key Cryptography, organized by Fischlin, May, Rabin, and Pointcheval, September 2016. Big Data has been one of the topics in the seminar.
\end{itemize}


\paragraph{Shift of Topic.}
The main focus of the project \emph{SecOBig} was to perform operations on large amounts of cryptographically secured data. In the second phase, with project \emph{\memoc} we turn the focus to the research question what Big Data scenarios actually mean for cryptographic strengths. The reason for this transition is twofold. First, it should allow a smoother collaboration within the priority program. Here, we especially refer to the projects \emph{Scalable Cryptography} of Hofheinz and Kiltz and \emph{Big-Data-DynAmO: Dynamic, Approximate, and Online Methods for Big Data} of Meyer. The former one touches related questions concerning tightness in cryptographic security proofs, and the latter one deals (among others) with limited memory resources. To best of our knowledge, both projects will be continued in the second phase of the priority program.

The other reason is based on recent developments in the cryptographic community. One is that, in the past years, there has been a growing trend to perform general secure operations via so-called garbled circuits. This approach shows impressive performances, but neither one of the applicants here is an expert on this. Also, with NIST's recent call for post-quantum secure primitives and the growing interest by companies like Google, looking into this area in the context of Big data processing seems to be a more fashionable and research-wise a more promising topic.


