

\paragraph{Workpackage \Alph{workpackage}.1: Memory-efficient Reductions}

\marc{to be done}

\paragraph{Workpackage \Alph{workpackage}.2}

\marc{to be done}

\stepcounter{workpackage}


\paragraph{Work package \Alph{workpackage}.1: Adaptation to TLS~1.2.}

In this work packages we adopt ideas from the TLS~1.2 concept \cite{TLS12} to the suggested LWE-based key exchange protocol designs. The main step is to investigate if one can remove the late signatures in \cite{BCNS15} which make the solution slightly TLS-non-conforming. As discussed in \cite{BCNS15} such a change would most likely imply to switch to a PRF-ODH like assumption, allowing the adversary to mount an active attack against the underlying LWE problem. As pointed out by Peikert \cite{P14}, the problem seems to be easy under such active attacks, though. 

Fortunately, not all is lost: For TLS~1.2 the analysis of Jager et al.~\cite{JKSS12} requires only a very limited form of active attacks in which the adversary can make a single chosen queries only. For such active attacks the LWE problem may still be hard. As an alternative, or second step, we consider other designs possibilities for the key exchange protocol, thwarting this problem by design.


\paragraph{Work package \Alph{workpackage}.2: Adaptation to TLS~1.3.}

This work packages looks into the possibility to adapt the ideas of previous LWE-based proposals to (the current draft of) TLS~1.3~\cite{TLS13}. Since TLS~1.3 will be fundamentally different from TLS~1.2 this at foremost requires us to check if the current solutions can be transferred at all. 

Next, we address the question if we can augment existing protocols by a 0RTT mode where one derives a fresh key without interaction by consulting previous communication data. As explained above, this presumably requires an even stronger PRF-ODH like assumption, where the adversary can make many active queries. Here the question which should be addressed is if one needs to make some restriction on the number of key exchange sessions in which material is re-used.



\paragraph{Work package \Alph{workpackage}.2: Analysis of Underlying Problems.}

In this work package we investigate the hardness of the underlying problem(s) proposed in Work packages \Alph{workpackage}.1 and \Alph{workpackage}.2. This investigation covers several aspects. First, we verify with the results of Work packages \marc{XXXX; link auf ein vorheriges WP} how hard the problem itself seems to be. Secondly, we try to relate the new problem (via memory-efficient reductions) to the standard LWE problem, or try to show that the problem is strictly stronger, by giving a black-box separation result.


\stepcounter{workpackage}

